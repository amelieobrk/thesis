\chapter{Abstract}

In today's digital age, cybergrooming poses a dangerous threat to the lives of children and young people and requires automatic technologies with high detection performance to support child protection. This work analyzes the integration of LIWC-2022 features \cite{pennebaker2022liwc} into BERT representations \cite{devlin2019bert} using a cross-attention-based feature fusion approach. The goal is to improve detection performance, while also enhancing explainability by identifying relevant LIWC features contributing to the model's decisions. On the basis of a dataset consisting of complete conversations from 605 identified sexual predators collected by the Perverted Justice Foundation and the  non-grooming dialogues by PAN12 \cite{inches2012pan}, a BERT baseline was created, which was additionally secured against domain and length leakage using synthetic data. Afterwards, the feature-fusion model was evaluated using two different LIWC-2022 feature sets. The first set included all 118 LIWC features, while the second set focused on a subset of 49 psychometric LIWC features that have been highlighted in the literature as relevant for manipulative communication in grooming chats. The results show, that feature fusion increased the F1 score by 1.7\% to 98.7 \% using both kind of feature sets where especially the precision increased leading to a lower rate of false positives. In addition to the feature-fusion evaluation, an analysis of the differences between grooming and non-grooming conversations was conducted using LIWC features, revealing strong differences in categories like cognitive processes, social processes and temporal orientation. Furthermore, SHAP \cite{lundberg2017shap} explainability analysis was performed to identify the most relevant LIWC features contributing to the model's decisions. The SHAP analysis revealed that LIWC features contribute to a percentage of around 9.66\% when using the full LIWC feature set and about 7.41\% for the psychometric subset to the model's predictions. Also, the SHAP analysis confirmed the relevance of the LIWC features related to cognitive and social processes when using the psychometric subset. Overall, this work highlights the potential of hybrid models that combine semantic understanding with psycholinguistic insights to better identify cybergrooming behavior in online communication.

%%% Überarbeitet