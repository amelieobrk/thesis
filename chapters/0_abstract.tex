\chapter{Abstract}

In today's digital age, cybergrooming poses a dangerous threat to the lives of children and young people and requires automatic technologies with high detection performance to support child protection. This work analyzes the integration of LIWC-2022 features \cite{pennebaker2022liwc} into BERT representations \cite{devlin2019bert} using a cross-attention-based feature fusion approach. The goal is to improve detection performance while enhancing explainability by identifying relevant LIWC features that contribute to the model's decisions. Based on a dataset consisting of complete conversations from 605 identified sexual predators collected by the Perverted Justice Foundation \cite{pj} and the non-grooming dialogues by PAN12 \cite{inches2012pan}, a BERT baseline was created, which was additionally secured against domain and length leakage using synthetic data. Afterwards, the feature-fusion model was evaluated using two different LIWC-2022 feature sets. The first set included all 118 LIWC features, while the second set focused on a subset of 49 psychometric LIWC features. The results show, that feature fusion increased the F1 score by 1.7\% to 98.7 \% using both kinds of feature sets where especially the precision increased, leading to a lower rate of false positives. In addition to the feature-fusion evaluation, an analysis of the differences between grooming and non-grooming conversations was conducted using LIWC features, revealing strong differences in categories according to cognitive processes, social processes and temporal orientation. Furthermore, SHAP \cite{lundberg2017shap} explainability analysis was performed to identify the most relevant LIWC features contributing to the model's decisions. The SHAP analysis revealed that, on average, LIWC features contributed approximately 9.66\% when using the complete LIWC feature set and about 7.41\% for the psychometric subset to the model’s predictions. Moreover, the SHAP analysis confirmed the relevance of the high-level LIWC dimensions \textit{Tone}, \textit{Authenticity}, \textit{Analytic Thinking}, and \textit{Clout} as well as the broader psycholinguistic domains \textit{Affect}, \textit{Cognition}, and \textit{Social Processes} for distinguishing between grooming and non-grooming conversations. In conclusion, this work highlights the potential of hybrid models that combine semantic understanding with psycholinguistic insights to better identify cybergrooming behavior in online communication.


%%geändert