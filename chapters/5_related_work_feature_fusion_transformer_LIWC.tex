%%%%% 1. Jia et al. (2023):
%%%Titel: DeepPsych: Enhancing Psychological Trait Prediction via Integrating LIWC and Deep Learning
%%Quelle: Proceedings of the AAAI Conference on Artificial Intelligence, 2023
%%%Inhalt: Das Paper präsentiert eine hybride Methode, bei der LIWC features in ein Transformer-basiertes Deep Learning-Modell integriert werden, um Persönlichkeitsmerkmale aus Social-Media-Texten vorherzusagen. Die Ergebnisse zeigen, dass die Kombination signifikant bessere Vorhersageleistungen (z. B. für Big Five) erzielt als traditionelle Verfahren.



%%%%% Decoding deception in the online marketplace: enhancing fake review
%detection with psycholinguistics and transformer models
%Joni Salminen1 · Mekhail Mustak2
%· Soon‑Gyo Jung3 · Hannu Makkonen1 · Bernard J. Jansen3




%%%%8. Lin et al. (2023) - "Identifying Anaclitic and Introjective Personality Styles from Patients' Speech"
%Quelle: arXiv (Computer Science > Computation and Language)

%Fokus: Automatische Inferenz von Persönlichkeitstypen aus Sprache bei Depression

%Klinische Anwendung: 79 Patienten mit Major Depression, klinische Diagnoseinterviews
%Methodik: LIWC + BERT-Features vs. traditionelle Fragebögen

%Hauptbefund: LIWC-basierte Klassifikation übertrifft fragebogenbasierte Ansätze signifikant

%Klinische Relevanz: Potential für automatisierte Persönlichkeitscharakterisierung in der Therapie



\begin{comment}

    Text Mining psychologischer Merkmale mittels LIWC“ von Alexandrowicz, Glasauer und Egger (2022) 


WICHTIG 
    Konkrete Aussagen zu Transformer-Modellen:
Im letzten Drittel des Dokuments (v. a. im Ausblick bzw. Diskussionsteil) heißt es sinngemäß:

LIWC sei regelbasiert und daher begrenzt, wenn es um die Erfassung semantischer Nuancen, Kontexte oder Ironie geht.

Um diese Grenzen zu überwinden, schlagen die Autoren vor, LIWC mit modernen Sprachmodellen (Transformern) zu kombinieren, um eine semantische Kontextualisierung zu ermöglichen.

Dabei verweisen sie auf die starken Leistungen von Transformer-Architekturen (wie BERT) in der automatischen Bedeutungserschließung, die es ermöglichen könnten, auch indirekte oder kontextabhängige psychologische Signale zu erfassen.

Diese Kombination aus symbolischen Features (LIWC) und distributiven, lernbasierten Repräsentationen (Transformer) wird als vielversprechender Hybridansatz für zukünftige Forschung bewertet.

Beispielhafte Formulierung im Originalstil (sinngemäß zusammengefasst):
Die Kombination regelbasierter Verfahren wie LIWC mit lernenden Systemen – etwa auf Basis neuronaler Sprachmodelle – könnte psychologische Textanalyse auf ein neues Niveau heben, indem semantische, ironische oder kontextabhängige Aspekte mit einbezogen werden
\end{comment}