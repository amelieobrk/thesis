\chapter{Conclusion}
The aim of this thesis was to analyze the integration of LIWC-2022 features into BERT representations using a cross-attention-based feature fusion approach to improve the detection of cybergrooming conversations in online chat logs. The goal was to enhance both the detection performance and the explainability of the model by identifying relevant LIWC features contributing to its decisions. It was shown, that the fusion of LIWC features with BERT representations improves the detection performance of cybergrooming conversations. Both the complete LIWC feature set and a psychometric subset led to improvements in F1 score, with a notable increase in precision and decrease in the false positive rate. Also, it was shown, that the model confidence in its predictions increased when integrating LIWC features into the model input. When looking at the most important LIWC categories distinguishing between grooming and non-grooming conversations, the LIWC categories according to \textit{Cognition} and \textit{Social} were particularly relevant. The SHAP explainability analysis supported these findings by ranking both categories among the top 20 contributing features. Moreover, SHAP revealed that already a subset of approximately 50\% of all LIWC features accounted for around 80\% of the model’s prediction strength, when token representations were held constant. In addition, the analysis of misclassifications provided insights into how certain conversational contexts or ambiguous linguistic patterns led to false positives and false negatives highlighting that, while the integration of LIWC features reduced the overall number of false positives, remaining errors often originated from subtle overlaps between grooming-like manipulation and ordinary social interaction. Integrating LIWC features into the baseline model therefore not only increased the detection performance but also gave strong insights into the psycholinguistic language patterns used in cybergrooming conversations and how they might influence model decisions. While this thesis also faced certain limitations, its results show the importance of approaching online grooming detection from an interdisciplinary perspective that includes linguistics, psychology, and computer science. These results may support the development of practical detection tools that combine linguistic and psychometric markers to enhance online safety. In conclusion, this work shows that combining BERT with psychometric features provides a promising step toward more robust and explainable approaches to grooming detection. 