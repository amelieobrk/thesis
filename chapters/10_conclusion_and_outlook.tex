\chapter{Conclusion}
The aim of this thesis was to analyze the integration of LIWC-2022 features into BERT representations using a cross-attention-based feature fusion approach to improve the detection of cybergrooming conversations in online chat logs. It was the goal to enhance the detection performance and the explainability of the model by identifying relevant LIWC features contributing to its decisions. For the LIWC analysis, the feature fusion was once performed and analyzed using the complete LIWC-2022 feature set and once using a psychometric subset of 49 LIWC features that have been highlighted in the literature as relevant for manipulative communication in cybergrooming chats. The evaluation was performed on a dataset consisting of complete conversations from 605 identified sexual predators collected by the Perverted Justice Foundation~\cite{pj} and non-grooming dialogues by PAN12~\cite{inches2012pan} with additional synthetic non-grooming data to secure the model from domain and length leakages. The results demonstrated that integrating LIWC based features with BERT representations enhanced the detection performance compared to the baseline model. Both the complete LIWC feature set and the psychometric subset showed improvements in the F1-score, accompanied by a notable increase in precision and a reduction in the false positive rate. Also, it was shown that the model confidence in its predictions increased when integrating LIWC features into the model input and that the inclusion of LIWC features led to label flips from positive to negative predictions, indicating that LIWC features helped the model avoid certain misclassifications. Furthermore, a SHAP explainability analysis was performed to identify how strongly the LIWC features contribute to the model's decisions and which LIWC features were the most relevant for the model's predictions. It was shown, that LIWC features contributed on average approximately 9.66\% when using the full LIWC feature set and about 7.41\% for the psychometric subset to the model’s predictions. The SHAP analysis revealed that especially the stylistic LIWC summary dimensions \textit{Tone}, \textit{Authenticity}, \textit{Analytic Thinking}, and \textit{Clout}, as well as the psycholinguistic macro categories \textit{Affect}, \textit{Cognition}, and \textit{Social Processes} were particularly relevant for distinguishing between grooming and non-grooming conversations. These findings are in line with previous research, which highlights the importance of these categories in manipulative communication in cybergrooming chats. Moreover, the SHAP analysis revealed that already a subset of approximately 50\% of all LIWC features accounted for around 80\% of the model’s prediction, when token representations were held constant. Finally, a missclassification analysis was performed to analyze how the LIWC feature distributions differed across True Positives, False Positives, True Negatives, and False Negatives. Based on the hypothesis that False Positives would resemble True Positives and False Negatives would be closer to True Negatives, the analysis confirmed the first assumption. False Positives indeed showed high similarity to True Positives. In contrast, the False Negatives showed less consistent behavior, having characteristics of both classes without a clear alignment toward either group. Integrating LIWC features into the baseline model, therefore, increased the detection performance and gave strong insights into the psycholinguistic patterns used in cybergrooming conversations and how they might influence model decisions. While this thesis also faced certain limitations, its results show the importance of approaching online grooming detection from an interdisciplinary perspective that includes linguistics, psychology, and computer science. In conclusion, this work shows that combining BERT with psychometric features provides a promising step toward more robust and explainable approaches to cybergrooming detection.