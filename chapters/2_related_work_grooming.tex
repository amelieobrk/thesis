\chapter{Related Work}

\section{Understanding Cybergrooming}
The terms “online grooming” and “cybergrooming” are often used to describe the process by which an adult establishes a digital relationship with a minor in order to achieve and facilitate subsequent sexual exploitation. In academic literature, “online grooming” is the internationally established term for this phenomenon, particularly in the fields of psychology, criminology and child protection. Nonetheless, the synonymous term “cybergrooming” is more commonly used in media education and German-language legal discourse, with an emphasis on the technological environment in which the grooming takes place~\cite{mladenovic2021cyber, schittenhelm2024cybergrooming}. \textbf{To maintain terminological clarity, the term “cybergrooming” is used throughout this thesis. } %geändert

Cybergrooming can be seen as a psychological manipulation process in which a perpetrator uses digital communication technologies to befriend a minor, initiate sexual interactions online, and/or arrange a physical meeting with the aim of sexual abuse~\cite{webster2021european}. This process usually takes place in several stages, in which trust is gradually built up, emotional dependence is created and the minor is desensitized to sexual content or requests~\cite{whittle2013review}. Groomers often present themselves as friendly and understanding, using engaging and emotionally supportive language rather than openly deceiving~\cite{broome2020psycholinguistic}. This process is facilitated by digital media, which allows perpetrators to remain anonymous and gain unrestricted access to children without direct adult supervision~\cite{mladenovic2021cyber}. %geändert

It is important to note that online grooming doesn't refer to the sexual abuse itself, but rather to the preparatory phase and psychological manipulation. The perpetrator uses targeted strategies to influence the victim's thoughts, emotions, and decision making processes in such a way, that they serve his own intentions~\cite{schittenhelm2024cybergrooming}. This grooming dynamic is not limited to online contexts. The broader term “grooming” refers to any deliberate strategy used by perpetrators to emotionally condition and exploit defenseless minors and is a significant mechanism in cases of sexual abuse~\cite{craven2006grooming}.
%geändert 

While there is agreement on the manipulative nature of cybergroming, researchers have developed various models to represent its dynamics, communication strategies and stages of development\parencite{kloess2014online}. These models have been proposed to conceptualize the process of cybergrooming and often describe it as a multi stage process involving various phases such as friendship formation, relationship building, exclusivity, risk assessment, sexual interaction and conclusion~\cite{oconnell2003typology,whittle2013review,kloess2014online,chiangandgrant2017online}. Importantly, research suggests that these phases do not necessarily occur in a fixed or linear order and that perpetrators may skip phases, return to earlier phases, or adapt their strategies depending on the victim and context~\cite{whittle2013review,chiangandgrant2017online,Joleby2021offender}. For example, Black et al.~\cite{black2015linguistic} conducted a content analysis of Facebook chat logs and identified linguistic patterns, which are characteristic of different phases of grooming and provide empirical evidence for the nonlinear nature of the grooming process.

%geändert
\section{Cybergrooming as a Process}

A frequently cited model comes from Rachel O’Connell~\cite{oconnell2003typology}, who identified the six consecutive phases \emph{friendship building}, \emph{relationship building}, \emph{risk assessment}, \emph{exclusivity}, \emph{sexuality} and \emph{conclusion}. These phases form a kind of “linear roadmap” for the grooming process. The communication strategies of the phases range from small talk and compliments to risk assessment and sexual hints or appointments. Although O'Connell's model has strongly influenced the conceptual understanding of grooming dynamics, it has been criticized for its limited empirical basis~\cite{broome2020psycholinguistic, LorenzoDus2019}. Furthermore, the assumption of a fixed sequence has been questioned by the following empirical findings, suggesting that grooming often develops with overlapping or cyclical phases~\cite{Joleby2021offender}. Recent sequence analyses show that perpetrators often use trust building actions, sexualization, and risk assessment instead in parallel than in a strict sequence~\cite{Ringenberg2024assessing}.

An alternative, linguistically based model was proposed by Lorenzo-Dus et al.~\cite{lorenzo2016understanding}. Based on the analysis of 24 chat logs from the online archive Perverted-Justice.com~\cite{pj}, the authors divide grooming into the three dynamic phases \emph{access}, \emph{approach} and \emph{entrapment}. In particular in the entrapment phase , linguistic strategies often overlap, making it difficult to draw clear boundaries between the phases. The authors identify four communication processes consisting of deceptive trust development, sexual gratification, isolation and compliance testing. These processes are realized through strategic language use like praise, desensitization, and personal disclosures. Politeness strategies in particular play a central role in manipulating the victim by simultaneously signaling control and trust. However, despite its depth, the study remains limited in its generalizability as it relies on a purely qualitative methodology and is based on lure chats with adult volunteers. More recent work shows that lure chats differ from real victim interviews in key aspects like threats, coercion, and intensity of risk questioning \parencite{Ringenberg2024assessing}.  %%% approved

Another influential model was developed by Williams et al.~\cite{williamsmodel}, who analyzed convicted sex offenders in New Zealand. Their three-stage typology, consitsting of \emph{relationship building}, \emph{sexual content} and \emph{assessment}, underlines the dynamics of early manipulation. According to Williams et al.~\cite{williamsmodel}, in the \emph{relationship building} phase, offenders adapt their language to that of adolescents in order to create a feeling of closeness. The \emph{sexual content} phase often begins subtly, for example through games or advice and ends with explicit offers. The \emph{assessment} phase serves to continuously evaluate risk and create an emotional profile of the victim.

A psychologically based extension is also provided by the \emph{Luring Communication Theory}, which was developed by Olson et al.~\cite{olsonluring2007} and originally designed for offline grooming but can also be applied to online contexts \parencite{Cano2014}. This model comprises three phases containing \emph{Deceptive Trust Development}, \emph{Grooming} and \emph{Physical Approach}. In the first phase, an emotional connection is established through personal exchange. This is followed by increasing sexualized communication, which finally transitions into a phase in which physical contact is initiated.

All of these phase models show, that cybergrooming follows a structured and linguistically intervened progression. Understanding the phases and their communication strategies provides a conceptual basis for detecting grooming behavior in digital communication. This understanding is essential for developing automated detection systems that can identify grooming attempts based on linguistic patterns and conversational dynamics. The following section therefore presents current research on the automated detection of cybergrooming, with a focus on machine learning approaches and the integration of psychological insights into machine learning models.

\section{Towards Automated Detection of Online Grooming}

The high volume of modern online communication makes manual monitoring of cybergrooming in online chats, social media and online games nearly impossible. Automated detection systems are therefore essential for filtering and categorizing suspicious dialogues in real time~\cite{hamm2025llms}.
To support the detection and reduce the burden on victims, current research is increasingly focused on developing automated systems capable of detecting grooming behavior in online conversations. These systems aim to flag suspicious interactions in real time, enabling protective measures in digital environments.

Over the past decade, a variety of approaches have emerged, that rely on classical machine learning, the extraction of linguistic and behavioral features, and modern deep learning techniques. Broadly speaking, these approaches can be divided into the three main categories \textit{Identification of sexual predators}, where the goal is to classify entire conversations as grooming or non-grooming, \textit{Modeling the grooming process}, where the focus is on identifying the phases and strategies of grooming and early detection of grooming, with the goal of identifying criminal intentions in the earliest stages of communication

\subsection{Sexual Predator Identification}
Villatoro-Tello et al.~\cite{villatoro2012two} proposed a two stage framework specifically aimed to identify sex offenders who engage in grooming behavior in online chat environments. Their approach first filters suspicious conversations based on lexical, stylistic and emotional characteristics and then classifies individual messages according to different grooming phases. Based on the PAN12~\cite{inches2012pan} dataset for identifying sexual offenders, their model captures the sequential and manipulative nature of offenders' communication. Broome et al.~\cite{broome2020psycholinguistic} supported this approach and conducted a review of machine learning methods for identifying sex offenders. They confirmed, that the most effective approaches are based on linguistic features and highlighted the growing importance of deep learning models, that can capture contextual and temporal dependencies. Both studies underscore the value of stage aware and linguistic modeling in the development of automated systems to detect cybergrooming, while also pointing to the ethical need of transparency and explainability in high risk applications.

%%%geändert

\subsection{Grooming Process Modeling}
Understanding cybergrooming as a sequential and manipulative process has led to many efforts to model its different phases. For example, Gupta et al.~\cite{gupta2012characterizingpedophileconversationsinternet} conducted one of the earliest qualitative studies, using the Perverted Justice dataset to analyze grooming behavior in chat logs. They based their work on O'Connell's~\cite{oconnell2003typology} six-phase model and manually labeled the perpetrators' messages accordingly. Their findings underscored the structured nature of grooming strategies and laid important groundwork for following modeling efforts. %%Approved

Building on this foundation, Cano et al.~\cite{Cano2014} operationalized Olson's Luring Communication Theory~\cite{olsonluring2007} by defining the three grooming phases \textit{Deceptive Trust Development}, \textit{Grooming} and \textit{Seeking Physical Approach} themselves. They used linguistic, psycholinguistic and discourse-based features, including LIWC~\cite{tausczik2010psychological} and a sentiment analysis, to classify messages into the three phases using SVMs. Their results showed, that discourse features were especially effective in describing the progression of grooming behavior.  %% Approved

%% Überarbeitet
\subsection{Early Grooming Detection}
Efforts to detect grooming at an early stage often focus more on classifying single messages rather than a complete conversation. One of these approaches is presented by Isaza et al.~\cite{Isaza2022classifying}, who developed a CNN-based model trained on the PAN12 dataset to classify short messages as grooming or non-grooming. Their architecture utilizes semantic features through pre-trained word embeddings using the Word2Vec algorithm and applies convolutional filters of different sizes to detect n-gram-like patterns. Although the model achieved a high recall rate, it suffered from low precision, which was caused by the class imbalance in PAN12. Nevertheless, it was proved useful for early intervention as it correctly marked a large number of true positives.  % Approved

Another relevant contribution comes from Schläpfer et al.~\cite{schlaepfer2022online}, who dealt with machine learning methods for the early detection of sex offenders in online chats. Using pre-trained language models and ensemble classifiers, their approach focused on identifying grooming behavior at the conversation level. Although their work did not involve manual annotation at the message level, it showed that linguistic signals embedded in early parts of conversations can be used to distinguish chats from sex offenders from harmless chats. Their findings show the potential of automated systems to detect cybergrooming in its early stages and contribute to the development of real-time safety measures. %%approved

%%geändert


\subsection{Machine Learning Approaches}

While early research focused on the classification of sexual predators and the modeling of phases, modern work reflects a broader methodological field, which is often motivated by the need for scalable and interpretable detection systems. In their review, An et al.~\cite{an2025cybergrooming} highlight, that machine learning approaches play a central role in the automated detection of grooming, primarily due to their ability to process large amounts of online text. 
From this perspective, Gunawan et al. ~\cite{gunawan2016detecting} analyzed the behavioral indicators by training a classifier on the Perverted Justice~\cite{pj} dataset. Features like message length, conversation turns and response time were used to capture conversation flow. Their results suggest that these features, combined with oversampling techniques, can serve as strong predictors of predatory intent.

To capture the subtleties of psychological manipulation, Cook et al.~\cite{cook2023protecting} introduced a hybrid annotation approach involving psychologists, who manually labeled over 6000 chat messages according to eleven predefined grooming strategies. These annotations were then used to train deep learning models, which are capable of detecting finely tuned manipulation tactics. The study not only demonstrated the effectiveness of such models, but also underlined the importance of human expertise in capturing contextual meaning, which further underscores the need for hybrid systems in automated cybergrooming detection.

%%% Überarbeitet

In addition, Leiva-Bianchi et al.~\cite{leiva2024meta} performed a comparison of machine learning classifiers using the PAN12~\cite{inches2012pan} dataset. They evaluated classical algorithms, ensemble methods, and neural networks using lexical features like bag-of-words, TF-IDF and N-grams. Ensemble approaches, in particular Random Forest and Gradient Boosting, proved to be the most robust across all evaluation metrics, highlighting the value of combining different weak learners for detecting grooming in chat logs.

Preuß et al.~\cite{preuss2021automatically} went one step further and presented a two-staged detection method, that combines convolutional neural networks with a multilayer perceptron. Their system was also trained on PAN12 and analyzed lexical patterns using CNNs and supplemented them with behavioral signals like mood, reaction time ,and message frequency. A manually created line-level annotation enabled conversation-level classification and the identification of semantically relevant messages, resulting in a strong performance on several detection tasks. Similarly, Hamm and McKeever~\cite{hamm2025llms} analyzed machine learning for grooming detection with a special focus on conversational tone. Using the PAN12 dataset, they compared traditional models like SVM with the large language model LLaMA 3.2~\cite{llamapaper} and showed, that positively toned and emotionally warm strategies are used more frequently by groomers and are also easier to detect than negatively phrased behaviors. By combining sentiment analysis based on DistilBERT with classifier comparisons, their results showed that large language models achieve higher F1 scores and were able to better capture fine linguistic patterns, further underscoring the value of psychological modeling approaches.

All of these studies highlighted the potential of machine learning methods in detecting grooming. However, as An et al.~\cite{an2025cybergrooming} point out, a key challenge remains the lack of interdisciplinary integration. Many models operate in isolation from psychological theories, limiting their ability to generalize to real scenarios and cover social science insights into the dynamics of grooming. Consequently, the following sections explore transformer-based language models and explore how psycholinguistic profiling tools like LIWC can be used to bridge the gap between behavioral understanding and algorithmic accuracy.


%%geändert