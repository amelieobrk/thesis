\chapter{Introduction} 



\begin{comment}
    So ist es das Ziel von Oberkirch (2025), sprachliche Embeddings von DeBERTa v3 mit LIWC-basierten psychometrischen Features zu kombinieren, um die Erkennungsleistung zu steigern und gleichzeitig die Vorhersagen besser erklärbar zu machen. Konkret wird ein zweistufiges Modell aufgebaut: Zunächst feintuned man DeBERTa auf den segmentierten Chats, extrahiert dann pro Segment einen Feature-Vektor aus LIWC-Kategorien (z. B. Werte für Emotion, Kognition, soziale Prozesse etc.), und fügt beide Vektoren zusammen, um sie gemeinsam einem Klassifikator zuzuführen. Erste Experimente zeigen, dass diese Feature-Fusion die Erkennungsgenauigkeit verbessert und Aufschluss darüber geben kann, welche psychologischen Signale der Grooming-Kommunikation vom Modell genutzt werden. Insgesamt gelten hybride Modelle, die linguistisches Kontextverständnis eines großen Sprachmodells mit psycholinguistischen Indikatoren verknüpfen, als Best Practice, um subtile manipulative Strategien besser zu erkennen. Sie ermöglichen es zudem, die Entscheidungen des Modells inhaltlich zu interpretieren (etwa wenn hohe Werte in LIWC–Angst oder LIWC–Sexualität zum Ausschlag für eine Klassifizierung beitragen).
\end{comment}


% Hier könntest du ein paar saubere Statistiken einbringen
In today's world, social media are an integral part of the lives of children and young people and offer extensive opportunities for global networking and the exchange of ideas. In addition to the great potential of the latest technologies, social media also pose many dangers, especially for children and young people. Cybergrooming represents a significant danger in this regard - whereby minors are often contacted and emotionally influenced by adults using communicative manipulation in order to initiate or prepare sexual contact.
% Motivation, Problemstellung, Zielsetzung, Forschungsfrage, Aufbau der Arbeit


