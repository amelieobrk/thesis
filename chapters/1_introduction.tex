\chapter{Introduction} 

In today's digital age, where children and young people are connected to digital platforms alongside adults, cybergrooming poses a significant threat. Cybergrooming can be defined as the process by which an adult befriends a young person online to establish online sexual contact and sometimes a physical meeting with them, to commit sexual abuse~\cite{webster2021european}. Since the interaction between the victim and the perpetrator initially takes place online and the amount of data digital communication makes it impossible to detect cybergrooming manually~\cite{hamm2025llms}, automated text analysis and detection systems play a significant role here. Especially in recent years, AI-supported systems have become established, which are designed to detect and report grooming based on text in online chats. However, it is very important to identify all the cases of cybergrooming as early as possible,  in order to prevent potential harm to the victims~\cite{vogt2021early} and, at the same time, avoid false alarms and incorrect interventions as much as possible. For this reason, it is necessary to make AI-supported decisions interpretable and understandable, so that alarms can always be triggered on a comprehensible basis.

In addition to the technical approach to detecting cybergrooming, numerous psychological theories have emerged that analyze perpetrator and victim behavior in the process of online cybergrooming communication~\cite{black2015linguistic, lorenzo2016understanding, oconnell2003typology}. Therefore, many linguistic patterns have been identified, which are used by predators to manipulate and motivate victims to meet up in person~\cite{chiangandgrant2017online, lorenzudus2017cause}. A wide range of researchers agree that cybergrooming occurs within a process that includes, among other things, building trust, creating a bond, and hinting at sexual acts~\cite{oconnell2003typology,lorenzudus2017cause,williamsmodel,chiangandgrant2017online}. This process contains corresponding linguistic strategies and psycholinguistic markers that can be used to identify cybergrooming in online communication~\cite{black2015linguistic,guo2023text,broome2020psycholinguistic}. These Psycholinguistic markers can be identified using the tool Linguistic Inquiry and Word Count (LIWC), which was developed by Pennebaker et al.~\cite{tausczik2010psychological} and is widely used in psychological research. LIWC analyzes text, based on a dictionary of words and categorizes them into various linguistic and psychological categories like emotional tone, cognitive processes, social processes, and more~\cite{pennebaker2022liwc}. The tool provides insights into the psychological state and communication style of individuals based on their language use. LIWC has been successfully applied to detect social and personality traits in language~\cite{tausczik2010psychological}. In the context of cybergrooming detection, LIWC features have additionally served as a strong foundation to distinguish between grooming and non-grooming communication~\cite{broome2020psycholinguistic,guo2023text,gupta2012characterizingpedophileconversationsinternet}. However, LIWC alone may not capture the full complexity of language use in cybergrooming communication as it does not consider the context and semantics of the text like transformer-based models do. Therefore, combining LIWC features with transformer-based language models like BERT is a promising approach to enhance the detection performance and interpretability of cybergrooming detection systems.

It is the goal of this work to analyze the integration of LIWC-2022 features into BERT representation using a cross-attention-based feature fusion approach. The goal is to improve detection performance, while also enhancing explainability by identifying relevant LIWC features that contribute to the model's decisions. Therefore, SHAP~\cite{lundberg2017shap} is used to quantify the contributions of both feature types to provide comprehensible and psychologically useful explanations for model decisions. It will be investigated, if the integration of LIWC features improves detection performance and the effects differ between the full LIWC feature set and a psychometric subset. Additionally, it will be analyzed, in which exact LIWC categories grooming and non-grooming conversations show the strongest differences and which contributed the most to the model's decisions.

This thesis is structured as follows: Chapter 3 introduces the theoretical background on cybergrooming research, transformer-based models of language classification, the LIWC-2022 tool and finally explainable-AI, all in the context of cybergrooming detection. In chapter 4, the dataset collection, as well as the required preprocessing steps, will be described. Chapter 5 presents the methodology of this thesis, including the BERT baseline model, the feature fusion model architecture,the SHAP explainability analysis and the analysis of missclassifications. In chapter 6, all results will be evaluated. Finally, Chapter 7 discusses the results and their meaning in context, including an analysis of LIWC feature differences between grooming and non-grooming conversations and an interpretation of the SHAP explainability results, as well as the limitations and possible future work of this thesis.

%geändert