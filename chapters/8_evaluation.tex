\chapter{Evaluation} \label{sec:evaluation}
This chapter presents the results of these evaluations, focusing on the performance improvements achieved through feature fusion compared to baseline models using only transformer embeddings or LIWC features alone.

\section{Initial BERT Finetuning Results}
\begin{table}[H]
\centering
\captionabove[Evaluation of initial BERT base model]{\textbf{Evaluation of BERT base model}}
\label{tab:bert_base}
\begin{tabular}{@{}lrrrrr@{}}
\toprule
Step (Epoch) & Loss & Accuracy & Precision & Recall & \textbf{F1} \\
\cmidrule(lr){2-6} 
3000 (0.52) & 0.007 & 0.998 & 0.999 & 0.998 & \textbf{0.999} \\
9000 (1.57) & 0.003 & 0.999 & 0.100 & 1.000 & \textbf{0.100} \\
15000 (2.61) & 0.004 & 0.999 & 0.999 & 1.000 & \textbf{0.100} \\
17169 (3.00) & 0.002 & 1.000 & 0.999 & 1.000 & \textbf{0.100} \\
\bottomrule
\end{tabular}
\end{table}
As shown in table \ref{tab:bert_base}, the initial BERT fine-tuning on the unmodified PJ and PAN12 datasets achieved \textbf{near-perfect performance, with an F1 score of 0.999 after only 3000 train steps}. This extremely high accuracy suggested that the model was likely exploiting dataset-specific artifacts rather than learning genuine grooming-related patterns. For example, all PJ conversations were labeled as grooming (1) and all PAN12 conversations as non-grooming (0), allowing the model to rely rather on superficial cues such as conversation length or stylistic differences between datasets than on semantic content. This motivated the development of a stricter preprocessing pipeline to mitigate domain leakage and length leakage. In the following sections the improved preprocessing, training and evaluation strategies will be evaluated in detail.

\newpage
\section{Bert Finetuning Results on different Chunk Sizes and Data Setups} \label{sec:bert_chunk_size_and_data_setup}

To evaluate the impact of chunk size and dataset splitting strategy on model performance, a series of experiments were conducted using three different chunk sizes (150, 250 and 512 tokens) and two dataset configurations: a \textit{separated split}, where synthetic non-grooming data was included only in the training set and a \textit{mixed split}, where synthetic data was present in both training and test sets. The chunk distribution and results for each configuration are summarized in the following tables.

\subsection{Fixed Chunk Size of 150 chunks}

\begin{table}[H]
\centering
\small
\caption[Evaluation for Chunk Size 150, Separated Split]{\textbf{Evaluation for Chunk size 150, separated split (synthetic data only in train).}}
\label{tab:150_separated}
\begin{tabular}{@{}lrrrr rr@{}}
\toprule
  & \multicolumn{5}{c}{\textbf{All}} 
  & \multicolumn{1}{c}{\textbf{Synth-only}} \\
\cmidrule(lr){2-6} \cmidrule(lr){7-7} 
Step (Epoch) & Loss & Accuracy & Precision & Recall & \textbf{F1} 
& Accuracy \\
\midrule
3000 (0.37) & 0.611 & 0.837 & 0.665 & 0.975 & \textbf{0.790} & 0.231  \\
9000 (1.11) & 0.555 & 0.859 & 0.696 & 0.981 & \textbf{0.814} & 0.303  \\
15000 (1.85) & 0.529 & 0.872 & 0.719 & 0.976 & \textbf{0.828} & 0.314  \\
21000 (2.59) & 0.498 & 0.888 & 0.747 & 0.973 & \textbf{0.845} & 0.408  \\
24315 (3.00) & 0.553 & 0.869 & 0.710 & 0.985 & \textbf{0.825} & 0.299 \\
\bottomrule
\end{tabular}
\end{table}

\begin{table}[H]
\centering
\small
\caption[Evaluation for Chunk Size 150, Mixed Split]{\textbf{Evaluation for chunk size 150, mixed split (synthetic data in train + test).}}
\label{tab:150_mixed}
\begin{tabular}{@{}l *{10}{r} @{}}
\toprule
  & \multicolumn{5}{c}{\textbf{All}} 
  & \multicolumn{4}{c}{\textbf{Real-only}} 
  & \multicolumn{1}{c}{\textbf{Synth-only}} \\
\cmidrule(lr){2-6} \cmidrule(lr){7-10} \cmidrule(lr){11-11}
Step (Epoch) 
& Loss & Accuracy & Precision & Recall & \textbf{F1}
& Accuracy & Precision & Recall & F1
& Accuracy \\
\midrule
3000 (0.36) & 0.352 & 0.939 & 0.870 & 0.992 & \textbf{0.927} & 0.939 & 0.876 & 0.992 & 0.930 & 0.940 \\
9000 (1.07) & 0.257 & 0.975 & 0.958 & 0.979 & \textbf{0.969} & 0.974 & 0.959 & 0.979 & 0.969 & 0.993 \\
15000 (1.79) & 0.254 & 0.977 & 0.968 & 0.974 & \textbf{0.971} & 0.977 & 0.970 & 0.974 & 0.972 & 0.987 \\
21000 (2.50) & 0.248 & 0.980 & 0.965 & 0.986 & \textbf{0.975} & 0.980 & 0.965 & 0.986 & 0.975 & 0.993 \\
25200 (3.00) & 0.255 & 0.979 & 0.957 & 0.989 & \textbf{0.973} & 0.978 & 0.958 & 0.989 & 0.973 & 0.991 \\
\bottomrule
\end{tabular}
\end{table}


\subsection{Fixed Chunk Size of 250 chunks}
\begin{table}[H]
\centering
\small
\caption[Evaluation for Chunk Size 250, Seperated Split]{\textbf{Evaluation for chunk size 250, separated split (synthetic data only in train).}}
\label{tab:250_separated}
\begin{tabular}{@{}lrrrr rr@{}}
\toprule
  & \multicolumn{5}{c}{\textbf{All}} 
  & \multicolumn{1}{c}{\textbf{Synth-only}}  \\
\cmidrule(lr){2-6} \cmidrule(lr){7-7} 
Step (Epoch) & Loss & Accuracy & Precision & Recall & \textbf{F1} & Accuracy \\
\midrule
3000 (0.54) & 0.567 & 0.854 & 0.672 & 0.969 & \textbf{0.793} & 0.158 \\
9000 (1.62) & 0.647 & 0.845 & 0.653 & 0.988 & \textbf{0.786} & 0.082 \\
15000 (2.69) & 0.608 & 0.858 & 0.673 & 0.991 & \textbf{0.802} & 0.145 \\
16704 (3.00) & 0.561 & 0.871 & 0.695 & 0.989 & \textbf{0.816} & 0.200 \\
\bottomrule
\end{tabular}
\end{table}

\begin{table}[H]
\centering
\small
\caption[Evaluation for Chunk Size 250, Mixed Split]{\textbf{Evaluation for chunk size 250, mixed split (synthetic data in test + train).}}
\label{tab:250_mixed}
\begin{tabular}{@{}l *{10}{r} @{}}
\toprule
  & \multicolumn{5}{c}{\textbf{All}} 
  & \multicolumn{4}{c}{\textbf{Real-only}} 
  & \multicolumn{1}{c}{\textbf{Synth-only}} \\
\cmidrule(lr){2-6} \cmidrule(lr){7-10} \cmidrule(lr){11-11}
Step (Epoch) & Loss & Accuracy & Precision & Recall & \textbf{F1}
& Accuracy & Precision & Recall & F1
& Accuracy \\
\midrule
3000 (0.52)  & 0.298 & 0.958 & 0.903 & 0.988 & \textbf{0.944} & 0.957 & 0.905 & 0.988 & 0.945 & 0.980 \\
9000 (1.57)  & 0.255 & 0.976 & 0.941 & 0.996 & \textbf{0.967} & 0.976 & 0.944 & 0.996 & 0.969 & 0.974 \\
15000 (2.61) & 0.242 & 0.984 & 0.964 & 0.993 & \textbf{0.978} & 0.984 & 0.965 & 0.993 & 0.979 & 0.991 \\
17247 (3.00) & 0.242 & 0.984 & 0.962 & 0.994 & \textbf{0.978} & 0.983 & 0.963 & 0.994 & 0.978 & 0.991 \\
\bottomrule
\end{tabular}
\end{table}


\subsection{Fixed Chunk Size of 512 chunks}
\begin{table}[H]
\centering
\small
\caption[Evaluation for Chunk Size 512, Separated Split]{\textbf{Evaluation for chunk size 512, separated split (synthetic data only in train). }}
\label{tab:512_separated}
\begin{tabular}{@{}lrrrr rr@{}}
\toprule
  & \multicolumn{5}{c}{\textbf{All}} 
  & \multicolumn{1}{c}{\textbf{Synth-only}} \\
\cmidrule(lr){2-6} \cmidrule(lr){7-7} 
Step (Epoch) & Loss & Accuracy & Precision & Recall & \textbf{F1} & Accuracy \\
\midrule
3000 (0.78) & 0.472 & 0.894 & 0.690 & 0.989 & \textbf{0.813} & 0.186 \\
6000 (1.55) & 0.517 & 0.883 & 0.668 & 0.987 & \textbf{0.797} & 0.092 \\
9000 (2.33) & 0.518 & 0.885 & 0.670 & 0.994 & \textbf{0.800} & 0.116 \\
11610 (3.00) & 0.534 & 0.885 & 0.670 & 0.994 & \textbf{0.801} & 0.093 \\
\bottomrule
\end{tabular}
\end{table}

\begin{table}[H]
\centering
\small
\caption[Evaluation for Chunk Size 512, Mixed Split]{\textbf{Evaluation for chunk size 512, mixed split (synthetic data in train + test).}}
\label{tab:512_mixed}
\begin{tabular}{@{}lrrrr rr@{}}
\toprule
  & \multicolumn{5}{c}{\textbf{All}} 
  & \multicolumn{1}{c}{\textbf{Synth-only}} \\
\cmidrule(lr){2-6} \cmidrule(lr){7-7}
Step (Epoch) & Loss & Accuracy & Precision & Recall & \textbf{F1} & Accuracy\\
\midrule
3000 (0.76) & 0.251 & 0.978 & 0.935 & 0.990 & \textbf{0.962} & 0.970 \\
6000 (1.52) & 0.282 & 0.970 & 0.907 & 0.995 & \textbf{0.949} & 0.954 \\
9000 (2.28) & 0.248 & 0.981 & 0.941 & 0.996 & \textbf{0.968} & 0.982 \\
11865 (3.00) & 0.245 & 0.982 & 0.945 & 0.996 & \textbf{0.970} & 0.982 \\
\bottomrule
\end{tabular}
\end{table}


The results of the different chunk sizes and split strategies show that \textbf{BERT generally achieves very high performance on the balanced dataset}, regardless of the chosen chunk size. In addition, the recall is significantly higher than the precision in all runs, indicating that the model correctly identifies almost all grooming cases, but produces some false positives. 

Another notable feature is that the model doesn't generalize well on synthetic data when it occurs exclusively in the test set. The accuracy in the “synth-only” column remains low in this setting, indicating a distribution shift between real and synthetic data. However, when the proportion of synthetic data is integrated into the training, the effect disappears almost completely and the model learns to successfully incorporate the synthetic data into its decision boundaries.

In addition, it can be seen that the marginal label baseline score increases slightly with increasing chunk size, which is probably due to greater stability of the classifications in longer contexts. Overall, the results confirm that BERT is able to recognize the grooming class almost perfectly, even with a larger context and in a balanced dataset, which is reflected in F1 scores of up to 0.97.

\textbf{Finally, it should be emphasized that the dataset is way more balanced than the original PAN12 dataset (33\% grooming cases vs. only 5\% in PAN12), which explains the excellent metrics. The high data balance leads to clearly separable classes, which further benefits the performance of the model.}


\subsection{Confusion Matrices for BERT Baseline across Chunk Sizes and Data Setups}
%%%%% Confusion Matrices + ROC Curves

\begin{figure}[H]
  \centering

  % --- Erste Reihe: nur Test-Synthetic ---
  \begin{subfigure}[t]{0.32\textwidth}
    \centering
    \includegraphics[width=\linewidth, trim={0 12 0 10},clip]{bert_baseline_plots/sep_synth_in_test/len150/confmat_epoch_03.png}
    \caption{Chunk size = 150 tokens}
  \end{subfigure}\hfill
  \begin{subfigure}[t]{0.32\textwidth}
    \centering
    \includegraphics[width=\linewidth, trim={0 12 0 10},clip]{bert_baseline_plots/sep_synth_in_test/len250/confmat_epoch_03.png}
    \caption{Chunk size = 250 tokens}
  \end{subfigure}\hfill
  \begin{subfigure}[t]{0.32\textwidth}
    \centering
    \includegraphics[width=\linewidth, trim={0 12 0 10},clip]{bert_baseline_plots/sep_synth_in_test/len512/confmat_epoch_03.png}
    \caption{Chunk size = 512 tokens}
  \end{subfigure}

  % --- Zweite Reihe: Train + Test-Synthetic ---
  \vspace{0.4cm}
  \begin{subfigure}[t]{0.32\textwidth}
    \centering
    \includegraphics[width=\linewidth, trim={0 0 0 10},clip]{bert_baseline_plots/mixed/len150/confmat_epoch_03.png}
    \caption{Chunk size = 150 tokens}
  \end{subfigure}\hfill
  \begin{subfigure}[t]{0.32\textwidth}
    \centering
    \includegraphics[width=\linewidth, trim={0 0 0 10},clip]{bert_baseline_plots/mixed/len250/confmat_epoch_03.png}
    \caption{Chunk size = 250 tokens}
  \end{subfigure}\hfill
  \begin{subfigure}[t]{0.32\textwidth}
    \centering
    \includegraphics[width=\linewidth, trim={0 0 0 10},clip]{bert_baseline_plots/mixed/len512/confmat_epoch_03.png}
    \caption{Chunk size = 512 tokens}
  \end{subfigure}

  \caption[Confusion matrices of BERT baseline after epoch 3.]{\textbf{Confusion matrices of the BERT baseline after epoch 3.}  
  Top row: models trained with synthetic data included only in the test set.  
  Bottom row: models trained with synthetic data included in both train and test set.  
  Each column corresponds to a different chunk length (150, 250, 512).}
  \label{fig:bert_confusionmatrices_epoch3}
\end{figure}

Figure \ref{fig:bert_confusionmatrices_epoch3} shows the confusion matrices for the BERT baseline across different chunk sizes and dataset configurations after epoch 3. When looking at the chunk lengths, the false positives and false negatives decrease steadily with an increasing chunk size for both, the false positive and false negative rate and the mixed and seperated data setup. This indicates that the model benefits from a larger context and higher word count, which is consistent with the improved metrics observed in table \ref{tab:bert_base}. It is evident, that the model with the synthetic data in both train and test set achieves a clearly better performance across all chunk sizes with a distingtly lower false positive and also a slightly lower false negative rate across all chunk sizes. This confirms the earlier observation that including synthetic data in training helps the model generalize better to this distribution, leading to fewer misclassifications. 

What is the most striking, is the general higher false positive rate compared to the false negative rate across all configurations. Especially in the setup with the synthetic data only in the test set, the false positive rate is notably higher than the false negative rate (e.g. 1601 FP vs only 18 FN for chunk length 512 tokens), particularly for smaller chunk sizes (e.g. 3864 FP vs 146 FN for chunk length = 150 tokens). Even in the mixed setup, where the performance is generally better, the false positive rate remains slightly higher than the false negative rate. This indicates that the model is generally more conservative in its predictions, preferring to classify uncertain cases as grooming rather than risking missing actual grooming conversations. This behavior is often desirable in practical applications where false negatives (missed grooming cases) can have serious consequences. As already mentioned, the dataset is very balanced, which also contributes to the low false negative rates. Overall, the confusion matrices confirm that BERT performs very well across all configurations, with performance improving with larger chunk sizes and when synthetic data is included in training and with the best performance for chunk size 512 and mixed data setup. \textbf{Note that the values for accuracy, precision and f1 scores are still lower for the mixed setup with chunk size of 512 tokens compared to the mixed setup with chunk size of 250 tokens in table \ref{tab:bert_base}.} This happens because the dataset for 512-token chunks contains fewer samples overall. While the absolute number of false positives and false negatives is lower, their relative share among predictions is higher, which reduces precision. At the same time, recall slightly improves since almost no true positives are missed.


\subsection{ROC Curves for BERT Baseline across Chunk Sizes and Data Setups}

%%%%%%% ROC CURVES

\begin{figure}[H]
  \centering

  % ---------- Top row: synthetic only in TEST ----------
  \begin{subfigure}[t]{0.32\textwidth}
    \centering
    \includegraphics[width=\linewidth]{bert_baseline_plots/sep_synth_in_test/len150/roc_epoch_03_zoom.png}
    \caption{Chunk length = 150}
  \end{subfigure}\hfill
  \begin{subfigure}[t]{0.32\textwidth}
    \centering
    \includegraphics[width=\linewidth]{bert_baseline_plots/sep_synth_in_test/len250/roc_epoch_03_zoom.png}
    \caption{Chunk length = 250}
  \end{subfigure}\hfill
  \begin{subfigure}[t]{0.32\textwidth}
    \centering
    \includegraphics[width=\linewidth]{bert_baseline_plots/sep_synth_in_test/len512/roc_epoch_03_zoom.png}
    \caption{Chunk length = 512}
  \end{subfigure}

  % ---------- Bottom row: synthetic in TRAIN & TEST ----------
  \vspace{0.45cm}
  \begin{subfigure}[t]{0.32\textwidth}
    \centering
    \includegraphics[width=\linewidth]{bert_baseline_plots/mixed/len150/roc_epoch_03_zoom.png}
    \caption{Chunk length = 150}
  \end{subfigure}\hfill
  \begin{subfigure}[t]{0.32\textwidth}
    \centering
    \includegraphics[width=\linewidth]{bert_baseline_plots/mixed/len250/roc_epoch_03_zoom.png}
    \caption{Chunk length = 250}
  \end{subfigure}\hfill
  \begin{subfigure}[t]{0.32\textwidth}
    \centering
    \includegraphics[width=\linewidth]{bert_baseline_plots/mixed/len512/roc_epoch_03_zoom.png}
    \caption{Chunk length = 512}
  \end{subfigure}

  \caption[Zoomed ROC curves after epoch 3 for different chunk lengths.]{\textbf{Zoomed ROC curves after epoch 3 for different chunk lengths (BERT baseline).}
  Top row: models evaluated with synthetic data included \emph{only in the test set} (\(\mathrm{False Positive Rate}\le 0.20\) and \(\mathrm{TPR}\ge 0.90\)). 
  Bottom row: models trained and evaluated with synthetic data in \emph{train \& test} and therefore shown with a \emph{tighter zoom} to expose finer differences (\(\mathrm{False Positive Rate}\le 0.05\) and \(\mathrm{TPR}\ge 0.95\)). 
  Each panel shows the zoomed range of the ROC curve to better reveal separation at low false-positive rates; the legend reports the pAUC for the full ROC.}
  \label{fig:roc_zoom_epoch3}
\end{figure}

Figure \ref{fig:roc_zoom_epoch3} shows the zoomed ROC curves for the BERT baseline across different chunk sizes and dataset configurations after epoch 3. The ROC curves were plotted zoomed for both configurations to highlight the differences more clearly, as the model already performed very well, especially for the mixed dataset, with small differences in the F1 score of < 1\% after epoch 3. \textbf{Note that the zoom levels differ between the two setups.For the setup with synthetic data only in the test set, the ROC curves are shown with a zoom on false positive rates ≤ 0.20 and true positive rates ≥ 0.90, highlighting the overall performance at moderately low FPRs where differences between chunk sizes are still relatively small. For the mixed data setup, a stronger zoom was applied (FPR ≤ 0.05, TPR ≥ 0.95) to emphasize finer distinctions between the models in the critical region, where even small errors become relevant.} 

When looking at the chunk lengths, it is evident that the ROC curves improve steadily with increasing chunk size for both configurations. This indicates that the model benefits from a larger context and higher word count, which is consistent with the improved metrics observed in table \ref{tab:bert_base} and the confusion matrices in figure \ref{fig:bert_confusionmatrices_epoch3}. The pAUC values also increase with chunk size, confirming that larger chunks lead to better overall discrimination between grooming and non-grooming conversations. When comparing the pAUC scores between the two configurations, it is clear that the models trained with synthetic data in both train and test sets achieve significantly higher pAUC values across all chunk sizes. This confirms the earlier observation that including synthetic data in training helps the model generalize better to this distribution, leading to improved performance across the entire ROC curve. Overall, the zoomed ROC curves confirm that BERT performs very well across all configurations, with performance improving with larger chunk sizes and when synthetic data is included in training, consistent with previous analyses.

\section{Comparing LIWC-2022 Macro Groups} \label{sec:global_liwc_analysis}


\begin{figure}[ht]
    \centering
    \includegraphics[width=0.90\textwidth]{groups_mean_pan12_vs_pj.png}
    \caption[Comparison of aggregated LIWC macro-groups]{Comparison of aggregated LIWC macro-groups between PAN12 (non-grooming) and PJ (grooming) over global conversations.}
    \label{fig:liwc_macro_groups}
\end{figure}


Figure~\ref{fig:liwc_macro_groups} compares the mean values between the LIWC scores of PAN12 (non-grooming) and PJ (grooming) conversations, aggregated into their Macro Groups.

The results indicate that linguistic features dominate both corpora, with PJ showing significant higher values. \textit{Note that the collected PJ conversations are generally longer and therefore containing more linguistic markers overall.} More interesting is, that the groups \textit{Time}, \textit{Cognition} and \textit{Social} stand out, where PJ conversations show a stronger presence of temporal references (often linked to future planning of meetings), cognitive processes and social markers. Also, the Category ``Emoji´´ hows no Liwc-Values for both Datasets as a result of the slang handling and data preprocssing steps and different kind of emoji usage in the time, the datasets were collected.

Overall, the group-level aggregation highlights the major shifts across all LIWC-dimensions, providing a perspective to the category-wise analysis. This confirms that grooming communication is  characterized not only by increased length and density of language, but also by a distinct emphasis on cognitive, temporal and social processes.


\section{Comparing LIWC Features between PJ and PAN12 on Full Conversations} \label{sec:liwc_global_analysis}

\begin{figure}[ht]
    \centering
    \begin{subfigure}[t]{0.48\textwidth}
        \centering
        \includegraphics[width=\linewidth]{liwc_top15_d_global.png}
        \caption{Top 30 LIWC categories across all LIWC-2022 features.}
    \end{subfigure}
    \hfill
    \begin{subfigure}[t]{0.48\textwidth}
        \centering
        \includegraphics[width=\linewidth]{liwc_top15_d_by_psychometrics.png}
        \caption{Top 30 LIWC-2022 features highlighted in the literature.}
    \end{subfigure}
    \caption[LIWC feature Comparison (PJ, PAN12) over Complete Conversations]{LIWC feature comparison between grooming (PJ) and non-grooming (PAN12) dialogues based on cohens $d$ \cite{cohen1988}. Positive values indicate higher feature usage in grooming.}
    \label{fig:liwc_global_analysis}
\end{figure}


In addition, the effect size \textbf{Cohen’s $d$} \cite{cohen1988} was computed to quantify group differences. Figure \ref{fig:liwc_global_analysis} (left) shows the top 30 LIWC features with the largest absolute effect sizes across all LIWC-2022 features, while figure \ref{fig:liwc_global_analysis} (right) focuses on the top 30 features from the psychometric subset highlighted in prior literature (Section \ref{sec:liwc-feature-selection}). For both plots, values over 0 indicate higher feature values in PJ-Grooming conversations, while values below 0 indicate higher values in PAN12 non-grooming conversations. Note, that effect sizes appear lower in the psychometric subset, since it is fully contained within the complete feature set and does not include the strongest differentiating features as shown in the left figure. 

Again, as shown in figure~\ref{fig:liwc_global_analysis} (left), PJ conversations are overall much longer (higher word counts), have longer sentences, and have overall more linguistically features like pronouns, verbs and function words. Since that is a strong confounder, for more content-related interpretations, these features should be ignored. When looking at the dialogical style, many functions words seem to be more prevalent in grooming conversations, which could be related to the more complex sentence structures and higher word counts. Also, the higher word count makes the pj conversations caputure a broader range of topics leading to more diverse linguistic markers.
What is noticeable is the strong presence of the feature \textit{focusfuture}, which reflects the grooming strategy of planning future meetings. Also, next to the linguistic features, the thematic references like \textit{family}, \textit{home} and \textit{affiliation} stand out, being higher present in grooming conversations from PJ than in PAN12, which could also be more consistent with the typical grooming narratives (e.g. asking, if the parents are home). Furthermore, the complexity in grooming conversations tends to be lower than in non-grooming conversations, as indicated by the lower \textit{big words}, \textit{Analytic} and \textit{Culture} score, which could be due to the sources that PAN12 was collected from (e.g. forums, chatrooms) which often containing computer-related and more complex language.

When focusing on the psychometric features (right), the differences between grooming and non-grooming conversations become more pronounced. It is noticeable, that grooming conversations show a higher values in features like \textit{Cogntition}, \textit{cognitive processes} and \textit{Allure} which could be caused by a content related reference to seduction and manipulation. Therefore grooming conversations are clearly distinguishable based on psychological strategies like building closeness (social/affilation), attraction (allure) and cognitive engagement (cognitive processes). Also, the strongest signal of grooming conversations lies in the feature \textit{discrep} (would, should, could), showing a higher presence of words which might be used in boundary testing, conditioning and suggestions. This is accompanied by slight positive effects for \textit{tentant} (hedging) and \textit{polite} (courtesy or relationship building). Additionally, the features \textit{Drives}, \textit{insight}, \textit{achieve} and \textit{emotion anxiety} are more prevalent in grooming conversations, which could be related to the manipulative strategies used by groomers to build trust and emotional connection.
It is evident, that the PAN12 Conversations contain higher values in the features \textit{sexual}, \textit{male/female}, \textit{physical}, \textit{swear} and \textit{Social/social behavior/prosocial}. This is likely due to the fact that the PAN12 dataset contains a considerable amount of sexually explicit but non-grooming conversations for which were included to improve model-robustness. \textbf{Because LIWC computes scores as relative proportions, the shorter PAN12 conversations, which often consist almost entirely of sexual content, produce inflated values in sexual-related categories. In comparison, the longer and more diverse PJ logs dilute these terms within a broader linguistic context, leading to lower proportional scores.} Therefore it should be considered, that the PAN12 dataset is not a perfect representation of non-grooming conversations, but rather a challenging counterbalance to the grooming data. Overall, the LIWC analysis confirms that grooming conversations are characterized not only by increased length and density of language, but also by a distinct emphasis on cognitive, temporal and social processes, as well as specific psychological strategies related to manipulation and relationship building. These insights provide a deeper understanding of the linguistic and psychological markers of grooming behavior, which can inform the development of more effective detection models.


\section{Comparing LIWC Features between PJ and PAN12 and the Synthetic Dataset} \label{sec:liwc_synthetic_comparison}
% Im Text
\begin{figure}[ht]
  \centering
  \begin{subfigure}[t]{0.48\textwidth}
    \includegraphics[width=\linewidth]{synthetic_comparison_pj.png}
    \caption{PJ vs. synthetic}
  \end{subfigure}\hfill
  \begin{subfigure}[t]{0.48\textwidth}
    \includegraphics[width=\linewidth]{synthetic_comparison_pan.png}
    \caption{PAN12 vs. synthetic}
  \end{subfigure}
  \caption[Top 30 LIWC Differences with Synthetic Baseline]{Top LIWC differences with synthetic baseline based on cohens $d$ \cite{cohen1988}.}
  \label{fig:liwc_synth_side_by_side}
\end{figure}



Additionally, Figure~\ref{fig:liwc_synth_side_by_side} (left) shows the comparison between PJ and synthetic non-grooming data. Large discrepancies are visible across several LIWC dimensions, including strong positive shifts for all LIWC-categories such as \textit{Big Words}, \textit{ppron} and \textit{you}, suggesting that the synthetic samples over-represent certain linguistic markers.  Figure~\ref{fig:liwc_synth_side_by_side} (right) shows the comparison between PAN12 and the synthetic data. Here, the synthetic data again diverges notably across all LIWC-Categories. This confirms that the synthetic data differs clearly from both real corpora. Including such data in training therefore \textbf{challenges the model to generalize beyond the specific statistical patterns of PJ and PAN12 alone.} While this reduces predictive performance on real data in isolation only marginally, it improves the model’s ability to handle out-of-distribution examples, making the overall model more robust. Still, when interpreting the LIWC features of the predicted grooming and non-grooming class a clear distinction between real and synthetic data should be done, since the synthetic data postpones the real distribution in many LIWC-Categories.


\section{Chunk-based LIWC Analysis} \label{sec:chunk_based_liwc_analysis}


%%%% Überarbeitet

\begin{figure}[ht]
    \centering
    \begin{subfigure}[t]{0.49\textwidth}
        \centering
        \includegraphics[width=\linewidth]{heatmap_topdiff_gobal.png}
        \caption{Top 15 LIWC categories on 512 chunk-level across all LIWC-2022 features.}
    \end{subfigure}
    \hfill
    \begin{subfigure}[t]{0.49\textwidth}
        \centering
        \includegraphics[width=\linewidth]{heatmap_topdiff_psycho.png}
        \caption{Top 15 LIWC-2022 features on 512 chunk-level highlighted in the literature.}
    \end{subfigure}
    \caption[LIWC Feature Comparison at Chunk Level]{LIWC feature comparison between grooming and non-grooming dialogues based on LIWC-2022 Features at chunk level.}
    \label{fig:liwc_chunked_analysis}
\end{figure}

In addition to an analysis of the overall conversations, an analysis of the LIWC features at the chunk level of 512 chunks (as used in the later feature fusion) was performed to analyze local variations across chats. 

Figure \ref{fig:liwc_chunked_analysis} shows a heat map of the top 30 mean features of grooming and non-grooming conversations (left) with the largest difference across all LIWC features and (right) with the largest difference across only the psychometric features highlighted in the literature. Rows represent PJ Grooming, Synthetic Non-Grooming and PAN12 Non-Grooming, while columns represent LIWC features. Color intensity encodes the mean feature value per source. The heatmaps show, that the LIWC features differ significantly througout all features and three sources at the chunk level, making it easier to understand how a model can already achieve high performance at this level.

When looking at the full feature set (left), it is noticeable, that the main differences between grooming and (synthetic) non-grooming chunks mainly lie on linguistic features like \textit{Word Count}, \textit{Words per Sentence}, \textit{function} and \textit{pronouns}, which are all more prevalent in grooming chunks. This confirms the earlier observation that grooming conversations tend to be longer and more linguistically rich, even at the chunk level. It can be seen, that the synthetic data have a very high word count balance out possible length leakage effects on chunk level. The differences in all other categories are less pronounced, but still visible. Furthermore, there are also features like \textit{Analytic}, \textit{Tone} and \textit{Cognition} showing, that besides the linguistic features, also content related features differ between grooming and non-grooming chunks on chunk-level.

When focusing on the psychometric features (right), the differences between grooming and non-grooming chunks become more pronounced. This could be due to the fact, that the scale of the psychometric features is smaller, making differences more visible since there is no feature like word count dominating the scale. It is noticeable, that grooming chunks show a values in features like \textit{Cogntition}, \textit{cognitive processes},\textit{Allure} and \textit{Social (processis)} which could be caused by a content related reference to seduction and manipulation.  Also, Therefore grooming conversations are clearly distinguishable based on psychological strategies like building closeness (social/affilation), attraction (allure) and cognitive engagement (cognitive processes). Also, features like \textit{polite}, \textit{(positive)emotion}, \textity{affiliation} and \textit{communication} show notable differences between the grooming and non-grooming chats suggesting that these psychological strategies are also reflected in local text segments. It is striking, that again \textit{sexual} shows a higher presence in the PAN12 non-grooming chunks, which is likely due to the fact that the PAN12 dataset contains a considerable amount of sexually explicit but non-grooming conversations. This confirms the earlier observation that the PAN12 dataset is not a perfect representation of non-grooming conversations, but rather a challenging counterbalance to the grooming data. When comparing the synthetic non-grooming data to the real non-grooming data, it can be seen, that the synthetic data again shows a very different pattern across all psychometric features, confirming that the synthetic data differs clearly from both real corpora. 

This chunk-level perspective complements the global analysis by showing that distinguishing features are not only visible when analyzing entire conversations, but also occur within short text chunks. While the bar charts of the global analysis in Figure \ref{fig:liwc_global_analysis} highlight markers such as word count, pronoun usage and function words, the heat maps show that even locally, grooming dialogues have higher values in many categories. In particular, the dominance of sexual terms becomes apparent at the chunk level, as shorter segments amplify their relative frequency, whereas these signals are diluted in complete conversations due to more diverse linguistic content. At the same time, psychometric dimensions such as cognitive and social processes remain dominant in both analyses, underscoring their theoretical relevance.

Taken together, these results suggest that grooming behavior can be detected in the dataset through both global stylistic patterns and local lexical cues, which explains why BERT already shows very strong performance at the finetuning at chunk level.


\section{Feature Fusion Evaluation}

\begin{table}[H]
\centering
\small
\caption[Evaluation: Feature Fusion with all LIWC-2022 Features]{\textbf{Evaluation: Feature Fusion with all LIWC-2022 Features}}
\label{tab:fusion_all}
\begin{tabular}{@{}lrrrrr@{}} % <-- 6 Spalten (l + 5x r)
\toprule
& \multicolumn{5}{c}{\textbf{Test Set Metrics}} \\
\cmidrule(lr){2-6}
\textbf{Step (Epoch)} & \textbf{Loss} & \textbf{Accuracy} & \textbf{Precision} & \textbf{Recall} & \textbf{F1} \\
\midrule
3000 (0.76)  & 0.234 & 0.985 & 0.959 & 0.989 & \textbf{0.974} \\
6000 (1.52)  & 0.231 & 0.987 & 0.962 & 0.994 & \textbf{0.978} \\
9000 (2.28)  & 0.218 & 0.993 & 0.983 & 0.992 & \textbf{0.987} \\
11865 (3.00) & 0.218 & 0.993 & 0.989 & 0.985 & \textbf{0.987} \\
\bottomrule
\end{tabular}
\end{table}


\begin{table}[H]
\centering
\caption[Evaluation: Feature Fusion with LIWC-2022 Subset]{\textbf{Evaluation: Feature Fusion with Subset of LIWC-2022 Features}}
\small
\label{tab:fusion_subset}
\begin{tabular}{@{}lrrrrr@{}} % <-- 6 Spalten
\toprule
& \multicolumn{5}{c}{\textbf{Test Set Metrics}} \\
\cmidrule(lr){2-6}
\textbf{Step (Epoch)} & \textbf{Loss} & \textbf{Accuracy} & \textbf{Precision} & \textbf{Recall} & \textbf{F1} \\
\midrule
3000 (0.76)  & 0.249 & 0.976 & 0.925 & 0.996 & \textbf{0.959} \\
6000 (1.52)  & 0.221 & 0.991 & 0.983 & 0.985 & \textbf{0.984} \\
9000 (2.28)  & 0.219 & 0.992 & 0.990 & 0.983 & \textbf{0.986} \\
11865 (3.00) & 0.218 & 0.993 & 0.989 & 0.986 & \textbf{0.987} \\
\bottomrule
\end{tabular}
\end{table}

The presented tables \ref{tab:fusion_all} and \ref{tab:fusion_subset} summarize the performance of the feature fusion model that combines BERT embeddings with LIWC-2022 features. Both configurations were evaluated at four training steps (3000, 6000, 9000 and 11865), corresponding to epochs of 0.76, 1.52, 2.28 and 3.00.

\subsection{Using all LIWC-2022 Features}

When using all LIWC-2022 features (Table \ref{tab:fusion_all}), the fusion model achieved an improvement of the F1 score in all training steps of the feature-fusion model, starting from 0.974 at 3000 steps and reaching up to 0.987 at 11865 steps. This indicates that the additional psycholinguistic information provided by the LIWC features helps the model better distinguish between grooming and non-grooming conversations. It is striking, that the precision increased in all training steps from 0.935 to 0.959 at 3000 steps and up to a total difference of 4.4 \% to 0.989 at 11865 steps. This suggests that the fusion model is more effective at reducing false positives, which is crucial in practical applications where misclassifying non-grooming conversations as grooming can have serious consequences. On the other hand, there was a small decrease in recall from 0.990 to 0.989 at 3000 steps and from 0.996 to 0.985 at 11865 steps, indicating that the model is slightly less sensitive in identifying all grooming cases. This leads to the conclusion, that the fusion model is more conservative in its predictions using additional LIWC features, which may be beneficial in reducing false alarms while still maintaining high overall accuracy.


\subsection{Using a Subset of Psychometric LIWC-2022 Features}

When using only a subset of psychometric LIWC-2022 features (table \ref{tab:fusion_subset}), the same trend of improvement in F1 score was observed, starting from 0.959 at 3000 steps and reaching up to 0.987 at 11865 steps. This indicates that even a targeted  selection of psycholinguistically relevant features can significantly enhance model performance. The precision also showed a substantial increase from 0.925 to 0.959 at 3000 steps and up to 0.989 at 11865 steps, similar to the results when using all features. This reinforces the idea that psychometric features are particularly valuable in improving the model's ability to correctly identify grooming conversations while minimizing false positives. However, the recall showed a slight decrease from 0.996 to 0.985 at 6000 steps and from 0.983 to 0.986 at 11865 steps, indicating a minor trade-off in sensitivity. Overall, the results suggest that even a focused set of psychometric LIWC features can provide benefits when fused with BERT embeddings, enhancing the model's robustness and reliability in detecting grooming behavior.


\subsection{Confusion Matrices for Feature Fusion Models}


\begin{figure}[H]
  \centering

  % -------- Top row: Feature fusion with ALL LIWC-2022 features --------
  \begin{subfigure}[t]{0.32\textwidth}
    \centering
    % \includegraphics[width=\linewidth,trim={0 10 0 28},clip]{fusion_all_epoch01_confmat.png}
    \includegraphics[width=\linewidth]{plots_feature_fusion_all/fusion_eval_epoch_1/confusion_matrix_all_thr_0.50.png}
    \caption{Epoch 1}
    \label{fig:ff_all_e1}
  \end{subfigure}\hfill
  \begin{subfigure}[t]{0.32\textwidth}
    \centering
    \includegraphics[width=\linewidth]{plots_feature_fusion_all/fusion_eval_epoch_2/confusion_matrix_all_thr_0.50.png}
    \caption{Epoch 2}
    \label{fig:ff_all_e2}
  \end{subfigure}\hfill
  \begin{subfigure}[t]{0.32\textwidth}
    \centering
    \includegraphics[width=\linewidth]{plots_feature_fusion_all/fusion_eval_epoch_3/confusion_matrix_all_thr_0.50.png}
    \caption{Epoch 3}
    \label{fig:ff_all_e3}
  \end{subfigure}

  \vspace{0.45cm}

  % -------- Bottom row: Feature fusion with PSYCHOMETRIC subset --------
  \begin{subfigure}[t]{0.32\textwidth}
    \centering
    \includegraphics[width=\linewidth]{plots_feature_fusion_subset/epch_1_plots/confmat_thr_0.50_epoch_1.png}
    \caption{Epoch 1}
    \label{fig:ff_psy_e1}
  \end{subfigure}\hfill
  \begin{subfigure}[t]{0.32\textwidth}
    \centering
    \includegraphics[width=\linewidth]{plots_feature_fusion_subset/epoch_2_plots/confmat_thr_0.50_epoch_2.png}
    \caption{Epoch 2}
    \label{fig:ff_psy_e2}
  \end{subfigure}\hfill
  \begin{subfigure}[t]{0.32\textwidth}
    \centering
    \includegraphics[width=\linewidth]{plots_feature_fusion_subset/epoch_3_plots/confmat_thr_0.50_epoch_3.png}
    \caption{Epoch 3}
    \label{fig:ff_psy_e3}
  \end{subfigure}

  \caption[Confusion matrices by epoch for feature-fusion models.]{\textbf{Confusion matrices by epoch for feature-fusion models (BERT baseline).}
  \emph{Top row:} fusion with all LIWC-2022 features. \emph{Bottom row:} fusion with the psychometric LIWC-2022 subset.
  Each panel shows true labels (rows) vs. predicted labels (columns); diagonal cells indicate correct predictions, off-diagonal cells are errors. The confusion matrices were generated using a classification threshold of 0.5.
  For fair visual comparison, keep the same color scale across all panels.}
  \label{fig:ff_confmats_epochs}
\end{figure}

Figure \ref{fig:ff_confmats_epochs} shows the confusion matrices for both feature-fusion configurations across three epochs. It is notable that both configurations achieve a very balanced performance, having very high true positive rates (TPR) and true negative rates (TNR), with stable values across all epochs, indicating excellent overall classification performance. The number of false positives (FP) and false negatives (FN) is very low in all cases, demonstrating that the models are effective at minimizing both types of errors. 

When comparing the two configurations, it is evident that the model using all LIWC-2022 features shows a slightly better performance in epoch 1, with fewer false positives and false negatives compared to the model using only the psychometric subset. However, as training progresses to epochs 2 and 3, the differences between the two configurations become negligible, with both achieving nearly identical performance in epoch 3. This suggests that while the additional LIWC features may provide some initial benefit, the psychometric subset is sufficient for achieving high accuracy once the model has been adequately trained. Consequently, the confusion matrices reveal that both models maintain high sensitivity and specificity across all three epochs of training and for both configurations (all LIWC features and only psychometric subset of features). This suggests that the feature-fusion approach effectively balances the trade-off between detecting grooming conversations and avoiding false alarms. Also, when comparing the feature fusion models to  the BERT-Baseline with 512 Chunks and a mixed split \ref{fig:bert_confusionmatrices_epoch3}, it is evident, that the feature fusion models achieve a smaller false positive rate, indicating a more "conservative" classification behavior, which is particularly important in practical applications where false alarms can have serious consequences. However, as already shown in table \ref{tab:fusion_all} and \ref{tab:fusion_subset}, there is a small increase in the false positive rate, indicating a minor trade-off in sensitivity. Still, the differences are relatively small, suggesting that while LIWC features provide additional value, the BERT embeddings already capture much of the relevant information. Overall, the results suggest that the feature-fusion approach is better in classifying non-grooming conversations directly but with a small increase in the false negative rate but an overall more balanced performance than the BERT baseline.


\subsection{ROC Curves for Feature Fusion Models}

%%%%%% ROC CURVES

\begin{figure}[H]
  \centering

  % -------- Top row: Feature fusion with ALL LIWC-2022 features --------

  % -------- Bottom row: Feature fusion with PSYCHOMETRIC subset --------
  \begin{subfigure}[t]{0.32\textwidth}
    \centering
    \includegraphics[width=\linewidth]{plots_feature_fusion_all/fusion_eval_epoch_1/roc_curve_all_zoom.png}
    \caption{Epoch 1}
    \label{fig:ffroc_psy_e1}
  \end{subfigure}\hfill
  \begin{subfigure}[t]{0.32\textwidth}
    \centering
    \includegraphics[width=\linewidth]{plots_feature_fusion_all/fusion_eval_epoch_2/roc_curve_all_zoom.png}
    \caption{Epoch 2}
    \label{fig:ffroc_psy_e2}
  \end{subfigure}\hfill
  \begin{subfigure}[t]{0.32\textwidth}
    \centering
    \includegraphics[width=\linewidth]{plots_feature_fusion_all/fusion_eval_epoch_3/roc_curve_all_zoom.png}
    \caption{Epoch 3}
    \label{fig:ffroc_psy_e3}
  \end{subfigure}

    \vspace{0.45cm}


    \begin{subfigure}[t]{0.32\textwidth}
    \centering
    \includegraphics[width=\linewidth]{plots_feature_fusion_subset/epch_1_plots/roc_zoom_epoch_1.png}
    \caption{Epoch 1}
    \label{fig:ffroc_all_e1}
  \end{subfigure}\hfill
  \begin{subfigure}[t]{0.32\textwidth}
    \centering
    \includegraphics[width=\linewidth]{plots_feature_fusion_subset/epoch_2_plots/roc_zoom_epoch_2.png}
    \caption{Epoch 2}
    \label{fig:ffroc_all_e2}
  \end{subfigure}\hfill
  \begin{subfigure}[t]{0.32\textwidth}
    \centering
    \includegraphics[width=\linewidth]{plots_feature_fusion_subset/epoch_3_plots/roc_zoom_epoch_3.png}
    \caption{Epoch 3}
    \label{fig:ffroc_all_e3}
  \end{subfigure}


  \caption[Zoomed ROC curves for feature-fusion models across epochs.]{\textbf{Zoomed ROC curves for feature-fusion models across all three epochs.}
  \emph{Top row:} fusion with all LIWC-2022 features. \emph{Bottom row:} fusion with the psychometric LIWC-2022 subset.
  Columns show epochs 1–3. All panels display the same zoom window to highlight differences at low false-positive rates
  (\(\mathrm{FPR}\le 0.05,\ \mathrm{TPR}\ge 0.95\)); the legend in each plot reports the pAUC of the \emph{full} ROC curve.}
  \label{fig:ff_roc_zoom_epochs}
\end{figure}

Figure \ref{fig:ff_roc_zoom_epochs} shows the zoomed ROC curves for both feature-fusion configurations across three epochs. It is notable that both configurations achieve very high pAUC values, indicating excellent overall discrimination between grooming and non-grooming conversations. Also, the curves for both configurations are very steep at a very low false positive rate, indicating that the models can achieve high true positive rates while keeping false positives very low. Especially in epoch 1, there is a light difference between the two configurations, with the model using all LIWC-2022 features showing a slightly better performance. However, as training progresses to epochs 2 and 3, the differences between the two configurations become negligible, with both achieving nearly identical, almost perfect performance in epoch 3. Consequently, the zoomed-in view reveals that both models maintain high true positive rates even at very low false positive rates, across all three epochs of training and for both configurations (all LIWC features and only psychometric subset of features). This suggests that the feature-fusion balances sensitivity and specificity effectively. Also, the pAUC lies between 0.987 and 0.994, showing stable performance across all epochs and configurations.

When comparing the ROC curves of the feature-fusion models to the BERT baseline in figure \ref{fig:roc_zoom_epoch3}, it is evident that the fusion models achieve a  better performance, particularly at very low false positive rates for both configurations with more stable and steep ROC-Curves. This is also true when using only a subset of psychometric LIWC features, indicating that the addition of LIWC features helps the model maintain high sensitivity while reducing false positives as already shown in the confusion matrices at Figure \ref{fig:ff_confmats_epochs}. However, the differences are relatively small since the differences are only visible at a very small FPR, suggesting that while LIWC features provide additional value, the BERT embeddings already capture much of the relevant information. \textbf{Overall, the ROC analysis confirms that feature fusion enhances model robustness and reliability in detecting grooming behavior, particularly in scenarios where minimizing false positives is critical.} 

\section{Ablation Studies based on SHAP} \label{sec:ablation_studies_shap}

The results of the ablation studies based on SHAP are presented in the following sections. The analysis focuses on understanding the contribution of individual LIWC features to the model's predictions and identifying the most influential features for distinguishing between grooming and non-grooming conversations. The results of the SHAP Analysis were once computed for only the "real" data (PJ + PAN12) to show their how they can be distinguished based on LIWC features and once for the complete dataset (PJ + PAN12 + synthetic) to show how the synthetic data influences the feature importance since the synthetic data was used during training to improve model robustness. To show the results of the LIWC-based SHAP Analysis, several steps were taken. First, the relative importance of LIWC features in comparison to text tokens was calculated based on their SHAP values (Table \ref{tab:liwc_vs_tokens}). Then, the cumulative curves of all features were plotted to show how many features are needed to explain a certain percentage of the model decision (Figure \ref{fig:cumulative_feature_importance_combined}). Next, a global ranking of the top 20 features was created based on their percentages of the total significance (Figure \ref{fig:global_feature_importance_combined}). The results were then evaluated by class and visualized according to the direction of the effect (Figure \ref{fig:feature_importance_by_class_combined}).

\subsection{Relative contribution of LIWC versus text tokens.}

Table~\ref{tab:liwc_vs_tokens} summarizes the relative contribution of LIWC features, based on SHAP values, in comparison to text tokens for the model’s predictions, averaged over $N = 1000$ explained test instances.  

When using the complete LIWC feature set, LIWC features accounted for approximately $9.66\% \pm 0.45\%$  of the model’s decision-making process on average, while text tokens contributed about $90.34\% \pm 0.45\%$.  

When restricting the LIWC input to the psychometric subset, the LIWC share decreased further to roughly $7.41\% \pm 0.28\%$, with tokens contributing $92.59\% \pm 0.28\%$.  

The relatively high standard deviations (14.30\% for the full LIWC set and 8.80\% for the psychometric subset) indicate that the LIWC contribution varies across individual predictions. This variation suggests, that while LIWC features play only a minor role overall, they can become locally influential for certain samples where psychological or linguistic cues are more pronounced in the text.

\begin{table}[h!]
\centering
\caption[Relative contribution of text tokens and LIWC features.]{Relative contribution (\%) of text tokens and LIWC features across $N = 1000$ explained test instances. For each feature set, the mean, median, standard deviation (std), standard error (SE), and 95\% confidence interval (CI) are shown.}
\label{tab:liwc_vs_tokens}
\begin{tabular}{lcccc}
\toprule
\textbf{Feature Set} & \textbf{Mean (±CI)} & \textbf{Median} & \textbf{Std} & \textbf{SE} \\
\midrule
\multicolumn{5}{l}{\textbf{LIWC Features}} \\
All LIWC Features     & 9.67 (8.78–10.55) & 6.73 & 14.30 & 0.45 \\
Psychometric Subset   & 7.41 (6.87–7.96)  & 5.63 &  8.80 & 0.28 \\
\addlinespace[0.3em]
\midrule
\multicolumn{5}{l}{\textbf{Text Tokens}} \\
All LIWC Features     & 90.34 (89.45–91.22) & 93.27 & 14.30 & 0.45 \\
Psychometric Subset   & 92.59 (92.04–93.13) & 94.37 &  8.80 & 0.28 \\
\bottomrule
\end{tabular}
\end{table}


\subsection{Cumulative Feature Importance}

To gain a better understanding of the cumulative importance of features, the cumulative feature importance plots for all the used LIWC features were created. These plots show how many features are needed to explain a certain percentage of the model decision.

\begin{figure}[H]
  \centering
  
  % Erste Reihe: mit synthetischen Daten
  \begin{subfigure}[t]{0.49\textwidth}
    \centering
    \includegraphics[width=\linewidth,height=6cm,keepaspectratio]{cumulative_importance_synth.png}
    \caption{All LIWC-2022 features (with synthetic data).}
    \label{fig:cum_synth_all}
  \end{subfigure}\hfill
  \begin{subfigure}[t]{0.49\textwidth}
    \centering
    \includegraphics[width=\linewidth,height=6cm,keepaspectratio]{cumulative_importance_psycho_synth.png}
    \caption{Psychometric subset (with synthetic data).}
    \label{fig:cum_synth_psycho}
  \end{subfigure}
  
  \vspace{0.5cm}
  
  % Zweite Reihe: ohne synthetische Daten
  \begin{subfigure}[t]{0.49\textwidth}
    \centering
    \includegraphics[width=\linewidth,height=6cm,keepaspectratio]{cumulative_importance_all_no_synth.png}
    \caption{All LIWC-2022 features (no synthetic data).}
    \label{fig:cum_no_synth_all}
  \end{subfigure}\hfill
  \begin{subfigure}[t]{0.49\textwidth}
    \centering
    \includegraphics[width=\linewidth,height=6cm,keepaspectratio]{cumulative_importance_psycho_no_synth.png}
    \caption{Psychometric subset (no synthetic data).}
    \label{fig:cum_no_synth_psycho}
  \end{subfigure}

  \caption[Cumulative importance of LIWC-2022 features.]{\textbf{Cumulative importance of LIWC-2022 features.} 
  Top row: with synthetic data. Bottom row: no synthetic data.}
  \label{fig:cumulative_feature_importance_combined}
\end{figure}

When looking at the curves of the cumulative feature importance (Figure \ref{fig:cumulative_feature_importance_combined}), it is noticeable that the top 20 features already explain a large portion of the model decision. This indicates that the model relies strongly on a few key features to make its predictions. The two curves for the psychometric subset (with and without synthetic data) show an almost identical progression, suggesting that the inclusion of synthetic data doesn't alter the relative distribution of feature importance. \textbf{Note, that the smaller size of the psychometric subset feature space leads to an overall higher importance per feature compared to the full feature set, where the importance is distributed over a larger number of features.}

In contrast, when considering the full LIWC-2022 feature set, clearer differences emerge between the two configurations. With synthetic data, the top 20 features already account for over 80\% of the model decision, whereas without synthetic data, they explain only slightly above 60\%. This shows, that the inclusion of synthetic data results in a stronger concentration of predictive power within a smaller number of highly influential features. Overall, the steep rise in both curves suggests that the model relies heavily on the most informative LIWC categories, while a large portion of the remaining features contributes very little to the overall prediction. This emphasizes that only a limited subset of LIWC dimensions captures most of the relevant psychological and linguistic signals used for classification. To gain a deeper understanding of which specific LIWC features are the most influential in the model's decision-making process, a global ranking of the top 20 features was created based on their percentage share of total importance, since these features collectively explain the majority of the model’s predictive behavior for both configurations. The results are presented in Figure \ref{fig:global_feature_importance_combined}.

\begin{figure}[H]
  \centering
  
  % Erste Reihe: mit synthetischen Daten
  \begin{subfigure}[t]{0.49\textwidth}
    \centering
    \includegraphics[width=\linewidth,height=6cm,keepaspectratio]{topN_share_importance_synth.png}
    \caption{All LIWC-2022 features (with synthetic data).}
    \label{fig:synth_all}
  \end{subfigure}\hfill
  \begin{subfigure}[t]{0.49\textwidth}
    \centering
    \includegraphics[width=\linewidth,height=6cm,keepaspectratio]{topN_share_importance_psycho_synth.png}
    \caption{Psychometric subset (with synthetic data).}
    \label{fig:synth_psycho}
  \end{subfigure}
  
  \vspace{0.5cm}
  

  \begin{subfigure}[t]{0.49\textwidth}
    \centering
    \includegraphics[width=\linewidth,height=6cm,keepaspectratio]{topN_share_importance_no_synth.png}
    \caption{All LIWC-2022 features (no synthetic data).}
    \label{fig:no_synth_all}
  \end{subfigure}\hfill
  \begin{subfigure}[t]{0.49\textwidth}
    \centering
    \includegraphics[width=\linewidth,height=6cm,keepaspectratio]{topN_share_importance_psycho_no_synth.png}
    \caption{Psychometric subset (no synthetic data).}
    \label{fig:no_synth_psycho}
  \end{subfigure}

  \caption[Top 20 LIWC features ranked by percentages of the total significance.]{\textbf{Top 20 LIWC features ranked by percentages of the total significance.} 
  Top row: with synthetic data. Bottom row: no synthetic data.}
  \label{fig:global_feature_importance_combined}
\end{figure}

The global rankings of the top 20 features (Figure~\ref{fig:global_feature_importance_combined}) confirm the observations from the cumulative curves. A small core of features contributes most to the model decision, with the type of features used differing between the full LIWC set and the psychometric subset. Note, that as already shown in the cumulative curves, the percentage feature importance per feature is generally lower for all LIWC features than in the psychometric subset. This can be explained by the generally larger feature set, which distributes the importance over a larger number of features that the model can rely on.

\paragraph{All LIWC features (with synthetic data).}
The results reveal a strong dominance of structural and stylistic indicators, with \textit{WC}, \textit{Tone}, \textit{Authentic}, \textit{Analytic}, and \textit{Clout} forming the top five features. These categories capture aspects of text length, emotional tone, authenticity, analytical reasoning and social confidence suggesting that the model not only relies on the main structural feature (Word Count) but also includes stylistic and psychological dimensions to differentiate between grooming and non-grooming dialogues. In addition, features such as \textit{Cognition} and \textit{Affect} also appear among the top 20, indicating that cognitive and emotional expressions still contribute meaningfully to the model’s predictions. The pronounced dominance of the top features aligns with the cumulative curve, where over 80\% of the model’s decision-making is explained by the top 20 features.

\paragraph{All LIWC features (without synthetic data).}
Without synthetic data, the same core features (\textit{WC}, \textit{Tone}, \textit{Authentic}, \textit{Analytic}, and \textit{Clout}) remain dominant, though their relative weights are more evenly distributed. This shows a broader reliance on multiple linguistic and psychological dimensions when only real data are used. While \textit{Affect} and \textit{Cognition} remain also among the relevant categories, their impact is comparatively weaker. Overall, it shows again, that the features are less concentrated on a few top indicators and instead have a wider range of stylistic and functional LIWC dimensions, consistent with the smoother cumulative importance curve.


\paragraph{Psychometric subset (with synthetic data).}
Here, the focus is clearly on cognitive, social and affective processes such as \textit{Cognition} (dominant), \textit{tone\_pos}, \textit{Social}, \textit{Physical}, \textit{cogproc}, \textit{Affect}, and \textit{insight}. In addition, features like \textit{comm}, \textit{socbehav}, \textit{tentat}, and \textit{cause} appear as secondary but consistent indicators. Compared to the configuration without synthetic data, the feature distribution is more top-heavy, with \textit{Cognition} showing a stronger dominance, suggesting that the model places greater emphasis on cognitive processes when synthetic samples are included.

\paragraph{Psychometric subset (without synthetic data).}
The ranking remains stable at its core (\textit{Cognition}, \textit{tone\_pos}, \textit{Social}, \textit{cogproc}, \textit{Affect}), but the importance is more evenly distributed across the top features. The stable dominance of cognitive and affective markers across both configurations suggests that reasoning, emotional tone, and social orientation are key discriminators between grooming and non-grooming dialogues.

To further gain insights into the direction of the effects of the individual features, the mean signed SHAP values were visualized in the following section.


\subsection{Singed Feature Importance by Class}

\begin{figure}[H]
  \centering
  
  % Erste Reihe: mit synthetischen Daten
  \begin{subfigure}[t]{0.49\textwidth}
    \centering
    \includegraphics[width=\linewidth,height=6cm,keepaspectratio]{topN_mean_signed_importance_synth.png}
    \caption{All LIWC-2022 features (with synthetic data).}
    \label{fig:synth_all_shap}
  \end{subfigure}\hfill
  \begin{subfigure}[t]{0.49\textwidth}
    \centering
    \includegraphics[width=\linewidth,height=6cm,keepaspectratio]{topN_mean_signed_importance_psycho_synth.png}
    \caption{Psychometric subset (with synthetic data).}
    \label{fig:synth_psycho_shap}
  \end{subfigure}
  
  \vspace{0.5cm}
  
  % Zweite Reihe: ohne synthetische Daten
  \begin{subfigure}[t]{0.49\textwidth}
    \centering
    \includegraphics[width=\linewidth,height=6cm,keepaspectratio]{topN_mean_signed_importance_all_no_synth.png}
    \caption{All LIWC-2022 features (no synthetic data).}
    \label{fig:no_synth_all_shap}
  \end{subfigure}\hfill
  \begin{subfigure}[t]{0.49\textwidth}
    \centering
    \includegraphics[width=\linewidth,height=6cm,keepaspectratio]{topN_mean_signed_importance_psycho_no_synth.png}
    \caption{Psychometric subset (no synthetic data).}
    \label{fig:no_synth_psycho_shap}
  \end{subfigure}

  \caption[Top 20 LIWC features ranked by mean signed SHAP value.]{\textbf{Top 20 LIWC features ranked by mean signed SHAP value.} 
  Top row: with synthetic data. Bottom row: no synthetic data. 
  Positive values indicate a shift toward the grooming class, negative values toward the non-grooming class. 
  \textit{Note: the scales differ between the plots. The left-hand plots (all features) have a larger range than the right-hand plots (psychometric subset).}}
  \label{fig:feature_importance_by_class_combined}
\end{figure}

Figure \ref{fig:feature_importance_by_class_combined} shows the top 20 LIWC features ranked by their mean signed SHAP values for both configurations (all LIWC-2022 features and psychometric subset) with and without synthetic data. Note, that the mean signed Feature plots show the same features as the global importance plots, but now colored by their direction of effect. Positive values indicate a shift toward the grooming class, negative values toward the non-grooming class. \textbf{However, the direction (positive = grooming, negative = non-grooming in the plots) does not mean that “high” values of a feature always have this effect.} Rather, it shows that deviations of the feature from the average contribute to the prediction in this direction. It is also striking, that the SHAP results indicate that several features change their direction when synthetic data are included, suggesting that the feature attribution is sensitive to the underlying data composition. This is particularly evident in the psychometric subset, where top features like \textit{Cognition} and \textit{tone\_pos} shift the direction of their effect when synthetic data is included. These changes in SHAP directionality are caused by differences in the explained sample set and background distribution, for example, whether synthetic data were included in the explanation phase or filtered out. \textbf{Since SHAP values are computed relative to a background distribution, even small changes in data composition can shift the baseline and therefore the average feature contributions.} 


\paragraph{Psychometric subset with vs. without synthetic data.}
In the psychometric space, the directional effects remain mainly consistent across both configurations, yet subtle shifts become apparent. Deviations associated with \emph{Cognition}, \emph{Social}, and \emph{Physical} processes tend to support the \textbf{grooming} class, whereas deviations related to \emph{positive tone} and \emph{cognitive processing} (\textit{cogproc}) more strongly indicate \textbf{non-grooming}. When synthetic data are included, the directions of certain features (for example: \textit{Cognition}, \textit{tone\_pos}) shift, indicating that the model’s attribution becomes more balanced and less polarized. Ambivalent categories such as \textit{insight} and \textit{socrefs} lose their strong directional bias, which aligns with the smoother distribution of feature importance seen in the cumulative curves. 


\paragraph{All LIWC features with vs. without synthetic data.}
When considering the full LIWC-2022 feature space, the directionality of the SHAP values reveals distinct behavioral patterns. In both configurations, \textit{WC} exerts the strongest positive influence toward the \textbf{grooming} class, indicating that longer or more elaborated text segments are characteristic of grooming-related conversations. \textit{Cognition}, \textit{Affect}, and \textit{Tone} also pull toward grooming, suggesting that cognitively engaged, emotionally expressive, and tonally intensified communication patterns are central to manipulative conversational strategies. On the other hand, the featurs \textit{Clout}, \textit{Authentic} and \textit{Analytic} constantly push toward the \textbf{non-grooming} class, indicating that higher social confidence, authenticity, and analytical reasoning are more typical for non-grooming dialogues. When synthetic data are included, these semantic-psychological indicators become more pronounced, while several linguistic structure features such as \textit{function}, \textit{OtherP}, and \textit{Analytic} partially reverse direction. This shift indicates that the model relies more strongly on meaningful psychological signals and less on formal proxies. Without synthetic data, the attribution is more heterogeneous. Stylistic and grammatical markers (\textit{function words}, \textit{pronouns}, \textit{punctuation}) gain relative weight, suggesting that the model draws on surface-level text patterns rather than on deeper psycholinguistic content.

Overall, the SHAP-based ablation studies reveal that while LIWC features contribute a relatively small portion to the overall model decision, they provide meaningful psychological and linguistic signals that enhance interpretability. The inclusion of synthetic data influences the concentration of feature importance and the directional effects, highlighting the sensitivity of feature attribution to data composition. Key cognitive, affective, and social dimensions consistently emerge as important discriminators between grooming and non-grooming conversations, underscoring their relevance in understanding manipulative communication patterns.


\section{Confidence Analysis and Label Flip Analysis} \label{sec:confidence_and_label_flip_analysis}

In addition to the performance metrics and feature importance analysis, further evaluations were conducted to analyze the model's confidence in its predictions and to analyze instances where the model's predictions changed between evaluations after using LIWC features als additional model Inputs. This analysis was conducted on the dataset including synthetic data to assess the overall behavior of the model with and without LIWC features. \textbf{For the analysis, a total of 6849 samples were evaluated to reduce computational effort.}

\begin{figure}[H]
  \centering
  
  % All features
  \begin{subfigure}[t]{0.49\textwidth}
    \centering
    \includegraphics[width=\linewidth,height=6cm,keepaspectratio]{confidence_analysis_plots/maxprob_scatter.png}
    \caption{Confidence shift analysis for all LIWC-2022 features.}
    \label{fig:confshift_all}
  \end{subfigure}\hfill
  % Psychometric subset
  \begin{subfigure}[t]{0.49\textwidth}
    \centering
    \includegraphics[width=\linewidth,height=6cm,keepaspectratio]{confidence_analysis_psycho_plots/maxprob_scatter.png}
    \caption{Confidence shift analysis for psychometric LIWC-2022 subset.}
    \label{fig:confshift_psycho}
  \end{subfigure}

  \caption[Confidence shift analysis with LIWC features.]{\textbf{Confidence shift analysis with LIWC features.} 
  Left: all LIWC-2022 features. Right: psychometric subset of LIWC-2022 features. 
  Both plots show model confidence with LIWC set to zero (x-axis) versus LIWC as-is (y-axis). 
  The diagonal indicates no effect. 
  Points above the diagonal correspond to increased confidence due to LIWC features, while points below indicate reduced confidence. 
  The farther a point lies from the diagonal, the stronger the influence of LIWC features on model confidence. }
  \label{fig:confidence_shift}

\end{figure}


When looking on figure \ref{fig:confidence_shift} it can been seen that for all LIWC-2022 features (left), a lot of samples lie above the diagonal, indicating that the model's confidence in its predictions increased when LIWC features were included. Also, most of the samples lie close to the upper border (0.9 - 1.0) indicating an overall really high confidence of the model in its predictions using all LIWC-2022 features.
This suggests that the additional information provided by the full LIWC feature set helps the model make more confident predictions in most of the cases. Note that there are also some points underneath the diagonal showing that in some cases the model confidence decreased when LIWC features were included. 

When looking on the psychometric subset of LIWC-2022 features (right), a similar pattern can be observed, but the effect is less pronounced. While there are still many points above the diagonal indicating increased confidence, the points are more widely spread and lie closer to the diagonal, suggesting that the impact of the psychometric subset on model confidence is more moderate compared to using all LIWC features. This indicates that while the psychometric features are still valuable, they may not provide as much additional information as the full feature set.

The following table \ref{tab:confidence_label_flip} additionally shows the results of the confidence and label flip analysis for both the complete LIWC-2022 feature set and the psychometric subset after analyzing 6849 samples.

\begin{table}[H]
\centering
\caption{Confidence and label flip analysis for all LIWC-2022 features vs. psychometric subset.}
\begin{tabular}{lcc}
\toprule
\textbf{Metric} & \textbf{All Features} & \textbf{Psychometric Subset} \\
\midrule
Number of samples ($n$)            & 6849    & 6849 \\
$\Delta \mu$ (mean)                & 0.1543  & 0.0219 \\
$\Delta \tilde{x}$ (median)        & 0.2363  & 0.0313 \\
$\Delta \sigma$ (std)              & 0.1481  & 0.0256 \\
$\Delta p_{10}$                    & -0.0166 & -0.0020 \\
$\Delta p_{90}$                    & 0.3081  & 0.0454 \\
Class 0 count                      & 4055    & 3942 \\
Class 1 count                      & 2794    & 2907 \\
Flipped predictions                & 156     & 9 \\
Flip rate                          & 0.0228  & 0.0013 \\
\bottomrule
\end{tabular}
\label{tab:confidence_label_flip}
\end{table}

Additionally the Agreeement matrices for the model with all LIWC-2022 features and the psychometric subset are shown in the following figure: 
\begin{figure}[H]
  \centering

  % All features
  \begin{subfigure}[t]{0.49\textwidth}
    \centering
    \includegraphics[width=\linewidth,height=6cm]{confidence_analysis_plots/confusion_predtrue_vs_predzero.png}
    \caption{Agreement matrix for all LIWC-2022 features.}
    \label{fig:agreement_all}
  \end{subfigure}\hfill
  % Psychometric subset
  \begin{subfigure}[t]{0.49\textwidth}
    \centering
    \includegraphics[width=\linewidth,height=6cm]{confidence_analysis_psycho_plots/confusion_predtrue_vs_predzero.png}
    \caption{Agreement matrix for psychometric LIWC-2022 subset.}
    \label{fig:agreement_psycho}
  \end{subfigure}

  \caption[Agreement matrices for predictions with vs. without LIWC features.]{\textbf{Agreement matrices for predictions with vs. without LIWC features.} 
  Left: all LIWC-2022 features. Right: psychometric subset of LIWC-2022 features. 
  Each matrix compares the predicted class with LIWC features (y-axis) against the prediction without LIWC features (x-axis). 
  Values on the diagonal indicate stable predictions, while off-diagonal values represent \emph{label flips}.}
  \label{fig:agreement_combined}
\end{figure}

When looking at table \ref{tab:confidence_label_flip}, it can be seen that the model with all LIWC-2022 features shows a much stronger confidence shift (mean increase of 0.1543) compared to the psychometric subset (mean increase of 0.0219). This indicates again, that adding the full LIWC feature set has a more pronounced effect on the models confidence in its predictions. The median values also reflect this trend, with a larger increase for the full feature set (0.2363) compared to the subset (0.0313). The standard deviation is higher for the full feature set (0.1481) than for the subset (0.0256), suggesting greater variability in confidence shifts when using all features which is also reflected when looking at the samples in figure \ref{fig:confidence_shift}. 

Also, the model with the complete LIWC-2022 Feature set seems to have a higher amount of label flip, flipping 156 prediction labels in total from Label Grooming to label Non-Grooming. In comparison, the model using only the psychometric subset of LIWC features has this kind of label flip. This indicates that the inclusion of all LIWC-2022 features leads to more changes in the model's predictions, which could be due to the additional information provided by a higher amount of LWIC-Features.

\textbf{Note, that there is only a label flip direction of Grooming to Non-Grooming, but not the other way around.} This suggests that the LIWC features help the model to be more conservative in its predictions, reducing false positives by reclassifying some grooming predictions as non-grooming. Also, when looking at the total amount of samples which were analyzed (6849 in total), the number of label flips is relatively small (2.28\% for all features and 0.13\% for the psychometric subset), indicating that the model's predictions are generally stable even when LIWC features are not included.

Overall, these results suggest that while both configurations enhance model confidence, the full LIWC-2022 feature set has a more substantial impact on both confidence shifts and prediction stability.


\section{Missclassification Analysis based on LIWC-Scores} \label{sec:misclassification_analysis}


\subsection{Results Summary: Full LIWC feature set}



\begin{figure}[H]
  \centering
  
  % Erste Reihe: TP vs FP und FP vs TN
  \begin{subfigure}[t]{0.49\textwidth}
    \centering
    \includegraphics[width=\linewidth,height=6cm,keepaspectratio]{liwc_fpfn_report_all_features/volcano_tp_vs_fp.png}
    \caption{TP vs.\ FP (all features).} % true positives + false positives ->
    \label{fig:volcano_all_tp_vs_fp}
  \end{subfigure}\hfill
  \begin{subfigure}[t]{0.49\textwidth}
    \centering
    \includegraphics[width=\linewidth,height=6cm,keepaspectratio]{liwc_fpfn_report_all_features/volcano_fp_vs_tn.png}
    \caption{FP vs.\ TN (all features).}
    \label{fig:volcano_all_tp_vs_tn}
  \end{subfigure}
  
  \vspace{0.5cm}
  
  % Zweite Reihe: FN vs TN und FN vs TP
  \begin{subfigure}[t]{0.49\textwidth}
    \centering
    \includegraphics[width=\linewidth,height=6cm,keepaspectratio]{liwc_fpfn_report_all_features/volcano_tn_vs_fn.png}
    \caption{FN vs.\ TN (all features).}
    \label{fig:volcano_all_fn_vs_tn}
  \end{subfigure}\hfill
  \begin{subfigure}[t]{0.49\textwidth}
    \centering
    \includegraphics[width=\linewidth,height=6cm,keepaspectratio]{liwc_fpfn_report_all_features/volcano_fn_vs_tp.png}
    \caption{FN vs.\ TP (all features).}
    \label{fig:volcano_all_fn_vs_tp}
  \end{subfigure}

  \caption[Volcano plots for all LIWC-2022 features.]{\textbf{Volcano plots for all LIWC-2022 features.} 
  Top row: TP vs.\ FP and TP vs.\ TN. Bottom row: FN vs.\ TN and FN vs.\ TP.}
  \label{fig:volcano_all_features}
\end{figure}

\subsubsection{Interpretation of Volcano Plots (All LIWC-2022 Features)}

Figure~\ref{fig:volcano_all_features} shows volcano plots for all LIWC-2022 features, comparing misclassified with correctly classified samples across all four groups (TP/TN/FP/FN). 
Each dot corresponds to a LIWC feature, with Cohen’s $d$ \cite{cohen1988} on the $x$-axis (direction and magnitude of effect size) and significance on the $y$-axis ($-\log_{10}$ of the FDR-adjusted $p$-value). The vertical dashed line at $x=0$ indicates no mean  difference between the groups, with features plotted to the right being more frequent in the first group and those to the left in the second group. 
The horizontal dashed line marks the significance threshold of FDR $=0.05$; points above this line represent features with statistically significant group differences. \textbf{Therefore Orange points indicate significant differences between the features of the groups with FDR<0.05.} Since the complete LIWC-2022 feature set is used, the number of points in each plot is the same (118) with varying numbers of significant features showing differences between compariosn groups.

\paragraph{Top row (TP vs.\ FP and TP vs.\ TN).}
The comparison of TP vs.\ FP (Figure~\ref{fig:volcano_all_features}a) shows a moderate number of significant LIWC features,  suggesting that false positives differ from true positives on selected LIWC dimensions, but overall remain relatively close. 
In contrast, the FP vs.\ TN comparison (Figure~\ref{fig:volcano_all_features}b) displays clearly more significant differences, indicating that false positives and true negatives have more distinct positions in LIWC space. 
Taken together, this supports the interpretation that FP samples are linguistically closer to TP than to TN, leading to their missclassification based on LIWC features.

\paragraph{Bottom row (FN vs.\ TN and FN vs.\ TP).}
For FN vs.\ TN (Figure~\ref{fig:volcano_all_features}c), some more significant features appear in comparison to the top row, suggesting measurable differences between these groups. 
FN vs.\ TP (Figure~\ref{fig:volcano_all_features}d) shows fewer significant points, indicating that false negatives are more similar to TP than to TN, although both comparisons reveal noticeable divergence. This suggests that FN samples have an intermediate position, but lean slightly closer to TP.

Taken together, the volcano plot patterns suggest that FP samples are more strongly aligned with TP, while FN samples occupy a more ambiguous space between TP and TN. Note, that the model trained on the total LIWC feature subset set achieves very high classification performance (F1=0.987), so the number of misclassifications is very low (only 96 out of 13,367 samples). This limits the statistical power of the comparisons, especially for FN (only 41 samples) and may explain the relatively small number of significant features in some contrasts. 


In addition table \ref{tab:liwc-misclass-all} summarizes key statistics from the misclassification analysis using all 118 LIWC-2022 features. It includes counts of true/false positives/negatives, centroid distances, numbers of significant features in each pairwise comparison, effect size statistics and proximity test results as an overview of the findings.


\begin{table}[H]
\centering
\caption{Summary of misclassification analysis with all 118 LIWC features.}
\label{tab:liwc-misclass-all}
\begin{tabular}{lcc}
\toprule
Metric & Value \\
\midrule
\# Samples (total) & 13,367 \\
TN / TP / FP / FN & 9,543 / 3,728 / 55 / 41 \\
Centroid dist.\ TN vs.\ TP (z-space) & 3.04 \\
\midrule
\textbf{FP vs.\ TN} \#sig (FDR<0.05) & 44 \\
Median $|d|$ of sig.\ features & 0.445 \\
Max $|d|$ of sig.\ features & 1.018 \\
Top-5 features (by $|d|$ \& FDR) & focusfuture, discrep, verb, i, Linguistic \\
\midrule
\textbf{FN vs.\ TP} \#sig (FDR<0.05) & 9 \\
Median $|d|$ of sig.\ features & 0.618 \\
Max $|d|$ of sig.\ features & 0.936 \\
Top-5 features & WC, politic, adj, BigWords, acquire \\
\midrule
\textbf{TN vs.\ FN} \#sig (FDR<0.05) & 19 \\
Median $|d|$ of sig.\ features & 0.352 \\
Max $|d|$ of sig.\ features & 0.764 \\
Top-5 features & focusfuture, function, Linguistic, male, ppron \\
\midrule
\textbf{TP vs.\ FP} \#sig (FDR<0.05) & 13 \\
Median $|d|$ of sig.\ features & 0.195 \\
Max $|d|$ of sig.\ features & 0.844 \\
Top-5 features & sexual, Comma, BigWords, WC, function \\
\midrule
\textbf{FN closer to TN than TP} & Prop.\ 0.463, $p_{\text{bin}}=0.73$ \\
\textbf{FP closer to TP than TN} & Prop.\ 0.873, $p_{\text{bin}}<10^{-8}$ \\
\bottomrule
\end{tabular}
\end{table}

\paragraph{Summary}
False positives are strongly aligned with true positives in LIWC space: nearly 87\% of FP samples are closer to the TP centroid than to the TN centroid, a result that is highly significant across all proximity tests. This pattern is reflected in the per-feature comparisons, where 44 LIWC features differ significantly between FP and TN with medium-to-large effect sizes. In contrast, false negatives do not consistently resemble true negatives: only 46\% are closer to the TN centroid and no statistical evidence supports a systematic shift toward TN. Per-feature differences between FN and TP exist for a small subset of features (9 significant), some with relatively strong effect sizes, but overall less consistent than for FP vs.\ TN. FN vs.\ TN contrasts also reveal some differences (19 features), suggesting partial but not decisive separation. Overall, misclassification patterns in the full LIWC space are dominated by the FP–TP alignment.



%%%%%%%


\subsection{Results Summary: Psychometric LIWC subset}

\begin{comment}


\begin{figure}[H]
  \centering
  
  % Erste Reihe: FP vs TN und FN vs TP
  \begin{subfigure}[t]{0.49\textwidth}
    \centering
    \includegraphics[width=\linewidth,height=6cm,keepaspectratio]{liwc_fpfn_report_psychometric/lollipop_fp_vs_tn.png}
    \caption{FP vs.\ TN (psychometric subset).}
    \label{fig:lollipop_psycho_fp_vs_tn}
  \end{subfigure}\hfill
  \begin{subfigure}[t]{0.49\textwidth}
    \centering
    \includegraphics[width=\linewidth,height=6cm,keepaspectratio]{liwc_fpfn_report_psychometric/lollipop_fn_vs_tp.png}
    \caption{FN vs.\ TP (psychometric subset).}
    \label{fig:lollipop_psycho_fn_vs_tp}
  \end{subfigure}
  
  \vspace{0.5cm}
  
  % Zweite Reihe: FP vs TP und FN vs TN
  \begin{subfigure}[t]{0.49\textwidth}
    \centering
    \includegraphics[width=\linewidth,height=6cm,keepaspectratio]{liwc_fpfn_report_psychometric/lollipop_tp_vs_fp.png}
    \caption{FP vs.\ TP (psychometric subset).}
    \label{fig:lollipop_psycho_fp_vs_tp}
  \end{subfigure}\hfill
  \begin{subfigure}[t]{0.49\textwidth}
    \centering
    \includegraphics[width=\linewidth,height=6cm,keepaspectratio]{liwc_fpfn_report_psychometric/lollipop_tn_vs_fn.png}
    \caption{FN vs.\ TN (psychometric subset).}
    \label{fig:lollipop_psycho_fn_vs_tn}
  \end{subfigure}

  \caption[Lollipop plots for psychometric LIWC features.]{\textbf{Lollipop plots for psychometric LIWC features.} 
  Top row: FP vs.\ TN and FN vs.\ TP. Bottom row: FP vs.\ TP and FN vs.\ TN.}
  \label{fig:lollipop_psycho_features}
\end{figure}




\end{comment}

\begin{figure}[H]
  \centering
  
  % Erste Reihe: TP vs FP und TP vs FN
  \begin{subfigure}[t]{0.49\textwidth}
    \centering
    \includegraphics[width=\linewidth,height=6cm,keepaspectratio]{liwc_fpfn_report_psychometric/volcano_tp_vs_fp.png}
    \caption{TP vs.\ FP (psychometric subset).}
    \label{fig:volcano_psycho_tp_vs_fp}
  \end{subfigure}\hfill
  \begin{subfigure}[t]{0.49\textwidth}
    \centering
    \includegraphics[width=\linewidth,height=6cm,keepaspectratio]{liwc_fpfn_report_psychometric/volcano_fp_vs_tn.png}
    \caption{FP vs.\ TN (psychometric subset).}
    \label{fig:volcano_psycho_tp_vs_tn}
  \end{subfigure}
  
  \vspace{0.5cm}
  
  % Zweite Reihe: FN vs TN und FN vs TP
  \begin{subfigure}[t]{0.49\textwidth}
    \centering
    \includegraphics[width=\linewidth,height=6cm,keepaspectratio]{liwc_fpfn_report_psychometric/volcano_tn_vs_fn.png}
    \caption{FN vs.\ TN (psychometric subset).}
    \label{fig:volcano_psycho_fn_vs_tn}
  \end{subfigure}\hfill
  \begin{subfigure}[t]{0.49\textwidth}
    \centering
    \includegraphics[width=\linewidth,height=6cm,keepaspectratio]{liwc_fpfn_report_psychometric/volcano_fn_vs_tp.png}
    \caption{FN vs.\ TP (psychometric subset).}
    \label{fig:volcano_psycho_fn_vs_tp}
  \end{subfigure}

  \caption[Volcano plots for psychometric LIWC features.]{\textbf{Volcano plots for psychometric LIWC features.} 
  Top row: TP vs.\ FP and TP vs.\ TN. Bottom row: FN vs.\ TN and FN vs.\ TP.}
  \label{fig:volcano_psycho_features}
\end{figure}

\subsubsection{Interpretation of Volcano Plots (Psychometric LIWC Subset)}

Figure~\ref{fig:volcano_psycho_features} shows volcano plots restricted to the 49-dimensional psychometric LIWC subset. The plot layout and axes are the same as in the previous section and can be interpreted  in the same way. The only difference is the number of points, which is now 49 in each plot, corresponding to the total number of psychometric LIWC features. Also, compared to the full feature set, the number of significant \textit{effects} is much smaller, which indicates that the psychometric dimensions have fewer differences between confusion groups.

\paragraph{Top row (TP vs.\ FP and FP vs.\ TN).}
The TP vs.\ FP comparison (Figure~\ref{fig:volcano_psycho_features}a) shows only a very small number of significant features, 
suggesting that false positives resemble true positives fairly closely in terms of the psychometric LIWC categories.  
In contrast, FP vs.\ TN (Figure~\ref{fig:volcano_psycho_features}b) produces the highest concentration of significant differences, 
highlighting that false positives are linguistically much closer to true positives than to true negatives in this reduced feature space.

\paragraph{Bottom row (FN vs.\ TN and FN vs.\ TP).}
The FN vs.\ TN comparison (Figure~\ref{fig:volcano_psycho_features}c) shows only one significant feature, 
while FN vs.\ TP (Figure~\ref{fig:volcano_psycho_features}d) reveals a similarly sparse pattern. 
This suggests that false negatives are difficult to distinguish from both true positives and true negatives using only the psychometric subset, 
although the relative number of significant effects still points to slightly more overlap with TPs than TNs.

\paragraph{Summary.}
Overall, the psychometric subset results in weaker separability compared to the full LIWC feature set. Again it has to be said, that the model trained on the psychometric LIWC feature subset also achieved a very high classification performance (F1=0.985), so that the number of misclassifications is very low (only 99 out of 13,367 samples). This limits the statistical power of the comparisons, especially for FN (only 40 samples) and may explain the relatively small number of significant features in some contrasts. The patterns still suggest that false positives align more closely with true positives than with true negatives, while false negatives occupy a more ambiguous position with limited distinctiveness in this reduced feature space. The following table \ref{tab:liwc-misclass-psych} summarizes key statistics from the misclassification analysis using the psychometric LIWC-2022 subset including the same metrics as table \ref{tab:liwc-misclass-all} above. 

\begin{table}[H]
\centering
\caption{Summary of misclassification analysis with psychometric LIWC subset (49 features).}
\label{tab:liwc-misclass-psych}
\begin{tabular}{lcc}
\toprule
Metric & Value \\
\midrule
\# Samples (total) & 13,367 \\
TN / TP / FP / FN & 9,539 / 3,729 / 59 / 40 \\
Centroid dist.\ TN vs.\ TP (z-space) & 1.54 \\
\midrule
\textbf{FP vs.\ TN} \#sig (FDR<0.05) & 10 \\
Median $|d|$ of sig.\ features & 0.411 \\
Max $|d|$ of sig.\ features & 0.836 \\
Top-5 features (by $|d|$ \& FDR) & discrep, allnone, allure, cogproc, Cognition \\
\midrule
\textbf{FN vs.\ TP} \#sig (FDR<0.05) & 1 \\
Median $|d|$ of sig.\ features & 0.430 \\
Max $|d|$ of sig.\ features & 0.430 \\
Top-5 features & family \\
\midrule
\textbf{TN vs.\ FN} \#sig (FDR<0.05) & 1 \\
Median $|d|$ of sig.\ features & 0.175 \\
Max $|d|$ of sig.\ features & 0.175 \\
Top-5 features & polite \\
\midrule
\textbf{TP vs.\ FP} \#sig (FDR<0.05) & 4 \\
Median $|d|$ of sig.\ features & 0.528 \\
Max $|d|$ of sig.\ features & 0.918 \\
Top-5 features & swear, allnone, mental, family \\
\midrule
\textbf{FN closer to TN than TP} & Prop.\ 0.55, $p_{\text{bin}}=0.32$ \\
\textbf{FP closer to TP than TN} & Prop.\ 0.661, $p_{\text{bin}}=0.009$ \\
\bottomrule
\end{tabular}
\end{table}

\paragraph{Summary (psychometric subset).}
Within the psychometric subset, the alignment of false positives with true positives remains visible but weaker than with the full LIWC feature set: about two thirds of FP samples are closer to the TP centroid than to TN, a result that is still statistically significant. The number of significant per-feature differences between FP and TN is smaller (10 features), though several show medium effect sizes. FN samples again show no consistent resemblance to TN, with proportions close to chance and non-significant tests. Only a single feature differentiates FN from TP and TN, indicating limited explanatory power of psychometric LIWC categories for FN errors. Overall, the psychometric subset captures some meaningful FP–TP alignment, but FN patterns remain largely unresolved.


\textbf{While this full-feature analysis provides an overview of differences between confusion groups, it also introduces substantial noise from dimensions that carry little explanatory value. To refine the evaluation, a focused analysis was therefore conducted on the Top-20 LIWC features, identified as most influential according to global SHAP values (Section~\ref{sec:shap_missclassification_analysis}). 
This allows for a clearer examination of whether misclassifications 
align with the patterns of the classes they were mistaken for.}


\section{Shap-based Analysis of Missclassification} \label{sec:shap_missclassification_analysis} 

\subsection{Total LIWC Feature Set}

\begin{figure}[H]
  \centering

  % FP plot (all features)
  \begin{subfigure}[t]{0.49\textwidth}
    \centering
    \includegraphics[width=\linewidth,height=8.5cm]{lollipop_plots_all/proximity_FP_topK.png}
    \caption{False Positives vs.\ True Positives/Negatives. The \textcolor{orange!70!black}{orange} dots indicate the mean positions of the false positives, with horizontal bars showing 95\% confidence intervals.}
    \label{fig:lollipop_fp_all}
  \end{subfigure}\hfill
  % FN plot (all features)
  \begin{subfigure}[t]{0.49\textwidth}
    \centering
    \includegraphics[width=\linewidth,height=8.5cm]{lollipop_plots_all/proximity_FN_topK.png}
    \caption{False Negatives vs.\ True Positives/Negatives. The \textcolor{purple!70!black}{purple points} indicate the mean positions of the false negatives, with horizontal bars showing 95\% confidence intervals.}
    \label{fig:lollipop_fn_all}
  \end{subfigure}

  \caption[Signed proximity plots for all LIWC-2022 features.]{\textbf{Signed proximity plots for all LIWC-2022 features.} 
  Left: False Positives relative to True Positives and True Negatives. 
  Right: False Negatives relative to True Positives and True Negatives. 
  Features are ordered by global SHAP importance. The x-axis is aligned with the global SHAP direction. Values to the right of zero indicate greater similarity to true positives, values to the left indicate greater similarity to true negatives. The purple/orange dots show the mean projected position of the misclassified samples.}
  \label{fig:lollipop_all}
\end{figure}


Figure \ref{fig:lollipop_all} shows the signed proximity plots for the Top-K LIWC features based on global SHAP importance when using all LIWC-2022 features. \textbf{The sample size for FP (55) and FN (41) is very small, which automatically leads to broader confidence intervals and makes the results less reliable. Therefore, the findings must be interpreted with caution, since broader intervals increase the probability of crossing the zero line.} It is also important to note, that the signed proximity plots \textbf{operate entirely in the SHAP-projected feature space. The orientation of features is determined by their SHAP contribution to the model decision (towards grooming vs. non-grooming), not by the raw feature values themselves.} Thus, the proximity analysis is valid as a model-internal interpretation, which shows whether misclassified samples behave more similarly to true positives or to true negatives in terms of their SHAP-based deviations. \textbf{It does not imply that higher or lower raw feature values directly correspond to grooming or non-grooming.}

Anyways, the direction of the projected means, where values greater than zero indicate greater similarity to true positives and values less than zero indicate greater similarity to true negatives, together with the aggregated proximity statistics in Table \ref{tab:proximity_key_results_all}, provide trends in the top-20 LIWC features with respect to misclassification. Consequently, the interpretation should focus mainly on the patterns across features, rather than on single features in isolation.

Two aspects are particularly relevant when interpreting the results:
\begin{itemize}
\item the position of the projected group means relative to zero (indicating similarity to TP vs. TN),
\item the proximity rate, showing the proportion of the top-20 features for which FP is closer to TP or FN is closer to TN.
\end{itemize}

Even with broad confidence intervals, the general directional trend remains visible, while the intervals themselves transparently reflect the uncertainty per feature.

When looking at the False Positives (left plot) in Figure \ref{fig:lollipop_all}, it becomes evident that the majority of the top LIWC features position themselves closer to the True Positive (TP) group. Specifically, nine of the twenty features have 95 \% confidence intervals lying entirely on the TP side of zero, whereas only two fall completely on the TN side. This pattern aligns with the high proximity rate of 0.9 reported in Table \ref{tab:proximity_key_results_all}, indicating that most false positives contain LIWC feature profiles similar to true positives. Among the most TP-aligned features are \textit{Analytic}, \textit{Clout}, \textit{Dic}, \textit{function}, \textit{verb}, \textit{Cognition}, \textit{WPS}, \textit{adj}, and \textit{cogproc}, which together represent cognitively and structurally driven dimensions of language showing that false positives share the same cognitively organized and linguistically structured profile as true positives, which often correspond to real grooming cases. In contrast, only a small subset of features like \textit{Linguistic} and \textit{Polite}, have confidence intervals completely on the TN-side. Therefore, the proximity analysis shows again, that false positives are closer to real grooming instances according to the full LIWC feature set, both in direction and in feature composition..

When looking at the False Negatives in figure \ref{fig:lollipop_all} (right), a more ambigous pattern  compared to the False Positives can be seen. The projected means span a range between –0.8 and +0.7, with most of the confidence intervals crossing the zero line. Only a few features show stable tendencies like \textit{WC},  \textit{Analytic} and \textit{function} lie completely on the TP side, while \textit{Linguistic} and \textit{OtherP} are completely TN-oriented, but on a lower rank than the ambigous top features. The generally wider confidence intervals reflect the smaller FN sample size (N=41) and further underscore the uncertainty. \textbf{Note, that the proximity rate in table \ref{tab:proximity_key_results_all} is only 0.25.} This shows that only in a quarter of the top-20 features, the means of the false negatives are closer to the means of the true negatives than to the true positives. When considering the SHAP ranking of features, it becomes clear that alignment in the higher-ranked dimensions is more informative than in the lower-ranked ones. Still, it stays difficult to draw a clear conclusion for the FN group. While the two of the five highest-ranked features ({WC} and \textit{Analytic}) show a tendency towards the TP side, most of the remaining features are highly ambiguous with confidence intervals crossing zero. This overall uncertainty makes it challenging to determine if FNs generally align more with TNs, even if some individual features suggest this trend.

Table \ref{tab:proximity_key_results_all} summarizes the key proximity statistics for the Top-K LIWC features based on global SHAP importance when using all LIWC-2022 features. 

\begin{table}[H]
\centering
\captionabove[Key proximity results (Top-K of all LIWC features)]{\textbf{Key proximity results for Top-K of all LIWC features.} The proximity rate was calculated by comparing the mean position of misclassified samples (FP or FN) to the means of the correctly classified groups (TP and TN) along the SHAP-oriented feature axis. The columns CI TP-side/TN-side/crosses-0 indicate the number of features (out of the Top-K) for which the 95\% confidence interval of the projected mean lies entirely on the TP side (greater than zero), entirely on the TN side (less than zero), or crosses zero (indicating uncertainty).}
\label{tab:proximity_key_results_all}
\begin{tabular}{@{}lrrrr@{}}
\toprule
Group & Proximity rate & CI TP-side (n) & CI TN-side (n) & CI crosses 0 (n) \\
\cmidrule(lr){2-5}
FP & 0.9 & 9 & 2 & 9 \\
FN & 0.25 & 3 & 2 & 15 \\
\bottomrule
\end{tabular}
\end{table}


\subsection{Psychometric LIWC Subset}



\begin{figure}[H]
  \centering

  % FP plot (psychometric features)
  \begin{subfigure}[t]{0.49\textwidth}
    \centering
    \includegraphics[width=\linewidth,height=8.5cm]{lollipop_plots_psycho/proximity_FP_topK.png}
    \caption{False Positives vs.\ True Positives/Negatives. The \textcolor{orange!70!black}{orange} dots indicate the mean positions of the false positives, with horizontal bars showing 95\% confidence intervals.}
    \label{fig:lollipop_fp_psycho}
  \end{subfigure}\hfill
  % FN plot (psychometric features)
  \begin{subfigure}[t]{0.49\textwidth}
    \centering
    \includegraphics[width=\linewidth,height=8.5cm]{lollipop_plots_psycho/proximity_FN_topK.png}
    \caption{False Negatives vs.\ True Positives/Negatives. The \textcolor{purple!70!black}{purple points} indicate the mean positions of the false negatives, with horizontal bars showing 95\% confidence intervals.}
    \label{fig:lollipop_fn_psycho}
  \end{subfigure}

  \caption[Signed proximity plots for the psychometric LIWC-2022 subset.]{\textbf{Signed proximity plots for the psychometric LIWC-2022 subset.} 
  Left: False Positives relative to True Positives and True Negatives. 
  Right: False Negatives relative to True Positives and True Negatives. 
  As before, features are ordered by global SHAP importance and the x-axis shows projected means on the SHAP direction where values greater than zero indicate greater similarity to true positives, while values less than zero indicate greater similarity to true negatives. Again, the purple/orange dots show the mean projected position of the misclassified samples.}
  \label{fig:lollipop_psycho}
\end{figure}

Figure \ref{fig:lollipop_psycho} shows the signed proximity plots for the Top-K LIWC features based on global SHAP importance when using the psychometric LIWC-2022 subset. Again, \textbf{the sample size for FP (59 ) and for FN (40) is very small, wich automatically makes the confidence intervals broader and the results in the plots and table less reliable.} 

When looking at the False Positives (left plot) in Figure~\ref{fig:lollipop_psycho}, a more heterogeneous picture emerges compared to the total LIWC feature set. Out of the 20 psychometric features, only five have 95\% confidence intervals fully on the TP side, while six lie entirely on the TN side and nine cross zero. This distribution is reflected in the proximity rate of 0.8 (see Table~\ref{tab:proximity_key_results_psycho}), showing that despite the smaller sample size, most False Positives still align more closely with the True Positives than with the True Negatives. Notably, the most influential feature overall \textit{Cognition} lies completely on the TP side, which might carry more weight in the interpretation. Next to \textit{Cognition}, the features \textit{cogproc}, \textit{socbehav}, \textit{sexual}, and \textit{food} also show a TP-aligned tendency, suggesting that misclassified non-grooming samples might often exhibit cognitively and socially engaged, affectively charged communication patterns similar to those of grooming instances. In contrast, features such as \textit{tone\_pos}, \textit{Social}, \textit{allure}, \textit{comm}, and \textit{discrep} lie fully on the TN side, reflecting more emotionally neutral or socially conventional tendencies. Overall, the pattern is less clear than in the complete LIWC feature set, where most of the top features aligned with the TP direction and the proximity rate reached 0.9. Here, the mixed directionalities and more zero-crossings indicate more variability in the psychometric space. Nevertheless, the overall proximity tendency still suggests that the False Positives in the psychometric feature space remain more similar to True Positives than to True Negatives but with less stability and distinctiveness than in the full feature set.


When looking at the False Negatives in Figure~\ref{fig:lollipop_psycho} (right), the pattern appears highly ambiguous. Almost all confidence intervals cross the zero line, indicating there is no directional tendency across the psychometric features. The projected means are distributed all around zero, with roughly half leaning toward the TP side and half toward the TN side. Only one feature \textit{female} lies completely on the TN side, while all others cross zero, showing that the psychometric cues of the False Negatives fluctuate around the decision boundary rather than aligning clearly with either class. This interpretation is supported by the proximity rate in Table~\ref{tab:proximity_key_results_psycho}, which shows a value of $0.35$, indicating that only about one third of the False Negative means lie closer to the True Negatives.  Together these results suggest, that the False Negatives don't exhibit a clear non-grooming profile but instead show mixed or weak psychometric tendencies that overlap with grooming-like communication. This suggests, that the misclassified samples are not simply "harmless" cases, that the model failed to detect, but instead contain psychometric signals partly overlapping with grooming-like communication. The high number of confidence intervals crossing zero further underlines that the psychometric features alone are insufficient to clearly separate those cases. Consequently, the observed False Negatives likely result from an interaction between weak psychometric cues and missing or subtle linguistic triggers, which highlights the need of integrating psychometric features with more linguistic or sequential indicators to improve model robustness.


\begin{table}[H]
\centering
\captionabove[Key proximity results (Top-K psychometric LIWC features)]{%
\textbf{Key proximity results for Top-K psychometric LIWC features.} The proximity rate was calculated by comparing the mean position of misclassified samples (FP or FN) to the means of the correctly classified groups (TP and TN) along the SHAP-oriented feature axis. The columns CI TP-side/TN-side/crosses-0 indicate the number of features (out of the Top-K) for which the 95\% confidence interval of the projected mean lies entirely on the TP side (greater than zero), entirely on the TN side (less than zero), or crosses zero (indicating uncertainty).}
\label{tab:proximity_key_results_psycho}
\begin{tabular}{@{}lrrrr@{}}
\toprule
Group & Proximity rate & CI TP-side (n) & CI TN-side (n) & CI crosses 0 (n) \\
\cmidrule(lr){2-5}
FP & 0.8 & 5 & 6 & 9 \\
FN & 0.35 & 0 & 1 & 19 \\
\bottomrule
\end{tabular}
\end{table}









