\chapter{Discussion}
In this chapter, the presented results will be discussed and interpreted in a larger context. The implications of the findings for real world applications will be considered, as well as the limitations of this thesis and potential future research. 

\section{Addressing Data Leakage and Shortcut Learning}
A central starting point was the unexpectedly strong baseline performance of BERT, which raised questions of potential data leakage and shortcut learning. Addressing this issue formed the methodological foundation for the following steps.

The near perfect baseline performance (F1 \approx 0.999) observed in initial BERT experiments was identified as a potential symptom of \textbf{data leakage, primarily due to domain and length artifacts between the PJ and PAN12 datasets.} To overcome this, the generation of synthetic non-grooming chats in PJ style with a higher length than the typical PAN12 chats was employed. By introducing synthetic non-grooming PJ chats, the model could no longer rely on source domain as a label proxy and by enforcing stricter train and test splits, chunk-based leakage across datasets was prevented. Also, label smoothing was applied to mitigate overconfidecne in its predictions, as well as an increased droput rate to reduce overfitting on the training data and therfore improve generalization to unseen data. To further analyze the impact of chunk lengths, the improved baseline was tested with a fixed padding on different chunk sizes, revealing that shorter chunks led to a \textbf{slight decrease in performance.} Still, the performance was really high across all three different kind of chunk sizes (512, 250, 150 tokens), indicating that the model could still learn relevant patterns even in smaller text segments. 
%%geändert
\subsubsection{Synthetic Data as a Countermeasure Against Leakage}
It was noticeable that including synthetic non-grooming Chats only in the test set, a strong drop in performance was visible, which highlights the lack of generalization, as well as a strong domain specific shortcut learning. \textbf{This showed the risks associated with shortcuts in models, particularly in security applications where robustness is critical. Nevertheless, questions regarding the generalizability of synthetic negative examples and potential residual artifacts remain open for further investigation.} When integrating the synthetic data in the train set, the performance drop disappeared, showing that the model could learn to generalize better with the synthetic data. It should be highlighted, that the synthetic data especially affected the precision, since the model achieved a much higher recall already without synthetic data, but the precision was much lower, indicating that the model made more false positive errors. Additionally, the SHAP analyses showed that, with synthetic data, the LIWC feature importance shifted away from formal proxies like length or punctuation towards semantically meaningful contrasts, for example, included in the macro LIWC groups \textit{Tone}, \textit{Authentic}, and \textit{Cognition}. This provides further evidence that the implemented leakage countermeasures were effective, as the model was forced to rely on meaningful signals rather than dataset-specific artifacts. Nevertheless, the use of synthetic data also highlights a limitation. While synthetic non-grooming chats could be generated without ethical concerns, the creation of synthetic grooming conversations with GPT was not feasible due to ethical and policy restrictions. Such data would have represented a much stronger counterbalance against domain leakage, as it could have provided alternative positive examples beyond the PJ dataset. Therefore, future work should explore ways to generate or collect ethically sourced positive examples to further enhance the model's robustness and generalization.
%%geändert
\section{Data Augmentation and Its Limitations in LIWC Analysis}

To further improve the model's robustness, it could be considered to additonally integrate \textbf{data augmentation techniques} like paraphrasing or backtranslation into the training data, to increase the diversity of the training data and help the model learn more robust features that are less sensitive to specific wording or phrasing. Nevertheless, in the present work, such methods were intentionally avoided, since \textbf{LIWC features are lexically defined and therefore highly sensitive to textual modifications~\cite{tausczik2010psychological}}. Artificial augmentation (for example synonym replacement or backtranslation) risks shifting the distribution of key categories (for example pronouns, affective terms, sexual language), which would reduce the validity of the following psychometric analyses. Such distortions would likely also affect SHAP explanations, as the method would attribute importance scores based on artificially altered inputs rather than on authentic linguistic patterns, thereby undermining interpretability. 

\section{Baseline Robustness}
Given that all three chunk sizes (150, 250 and 512 tokens) had a consistently strong performance, especially when synthetic data was integrated into the training set, the subsequent feature fusion was conducted with the 512-token mixed configuration. This setup provided the richest conversational context and thereby maximized the coverage of LIWC categories within each chunk, offering the most informative basis for integration with transformer representations and allowing for a more comprehensive analysis of psychometric features with SHAP. The overall strong performance across all settings could be partly attributed to the use of balanced training data, as prior work has shown that transformer-based models like BERT are highly sensitive to class imbalance and achieve more stable and reliable results under balanced conditions~\cite{henningnlpclassimbalance2023}. Importantly, the balancing was required in this work to enable stable SHAP analyses and ensure that the derived explanations were not dominated by class imbalance effects~\cite{liu2022balancedbackgroundexplanationdata,chen2024interpretable}. 


\section{Cross-Attention Fusion of LIWC and Transformer Representations}

The integration of psycholinguistic features from LIWC into a BERT-based model for online grooming detection has showed clear improvements in model performance and decision stability. The proposed feature fusion architecture integrated LIWC features with late fusion at layers 6 and 12 using cross-attention with gating. This design aligns with results from multimodal fusion research, which suggest that late fusion is effective for heterogeneous modalities and avoids the risks of overloading early linguistic representations~\cite{shankar2022progressivefusion}. Furthermore, attribution studies show, that late fusion improves interpretability, since SHAP attributions can more clearly disentangle the contributions of linguistic tokens and auxiliary features~\cite{shapcat2024interpretable}. The performance of the feature fusion model indicates that an integration of LIWC features at layers 6 and 12 was particularly effective in this setting, but the optimal integration depth is likely backbone-specific and requires further investigation. Also, the modular nature of the proposed mechanism makes it transferable to other transformer architectures like RoBERTa~\cite{liu2019roberta} or DeBERTa v3~\cite{he2023debertav3}, which are known to be more powerful than BERT. Nevertheless, it should be noted, that these models were trained on a larger corpus, which may include the PAN12 dataset, potentially leading to data leakage if used as a baseline. Therefore, future work should carefully evaluate the use of these models in this context.
%%geändert
\section{Performance Gains from LIWC Integration (Full Set vs. Psychometric Subset)}

The integration of LIWC features improved the F1 score to approximately 0.987 for the full feature set and the psychometric subset after three epochs of training, with a \textbf{particularly notable increase in precision, which leads to fewer false positives.} As the confusion matrices in figure \ref{fig:ff_confmats_epochs} showed, the feature fusion model led to a more balanced trade-off between precision and recall, with a reduction in False Positives. This balance of precision and recall is especially relevant in practical applications as it reduces the risk of false alarms, which can lead to unnecessary interventions while keeping the false negative rate low. 

It was shown that the full LIWC feature set provided slightly stronger effects in the feature fusion performance and later explainability analysis, while the psychometric subset offered a more streamlined approach with nearly equivalent performance. This has implications for deployment and complexity, suggesting that a reduced feature set may be preferable in resource-constrained environments without substantial loss in effectiveness. Still, the psychometric subset allowed an analysis of the most relevant psycholinguistic categories for online grooming detection, especially when combined with SHAP explanations, which deepened the focus of the analysis beyond only linguistic features. Importantly, once length and domain leakage were mitigated through the use of synthetic data, the analyses revealed that psycholinguistic categories gained weight relative to formal features. In particular, \textit{Cognition}, \textit{Affect}, \textit{Perception} and stylistic markers (for example \textit{Tone}, \textit{Clout}, \textit{Authenticity}, \textit{Analytic}) consistently emerged among the strongest predictors. Also, SHAP analyses across the full LIWC set and the psychometric subset revealed an overlap in the most influential psychometric categories, especially including \textit{Cognition} and \textit{Tone}. These features always appeared as strong indicators to distinguish between grooming and non-grooming behavior regardless of the feature space, implying that a core set of psychometric markers can carry much of the discriminative signal. At the same time, the full LIWC configuration highlights additional linguistic proxies like \textit{function words} and \textit{Analytic} style, which \textbf{might indirectly reflect psychological processes.} This shows that a compact psychometric core set could be sufficient for interpretability focused applications while the extended feature set offers additional cues that may further strengthen the classification in practice.
%%geändert
\section{Stabilizing Effects of LIWC on Model Predictions}

Based on an analysis of 3000 test samples (Table\ref{tab:liwc_vs_tokens}), it was shown that \textbf{the mean contribution of LIWC features amounts to approximately 9.66\% when using the full LIWC feature set and about 7.41\% for the psychometric subset.} Still, the integration of LIWC features has  been shown to enhance model confidence and performance. By providing additional context for the models predictions, LIWC features helped to reduce uncertainty in the decision-making process. This was shown by an increase in the mean confidence shift (\Delta \mu \approx 0.1543) when LIWC features were included, compared to a much smaller shift (\Delta \mu \approx 0.0219) when only the psychometric subset was used. The low rate of label flips (2.28\% for the full set and 0.13\% for the subset) further underscores the direction of the effect when integrating LIWC, with only changes being conservative reclassifications from grooming to non-grooming. This indicates that LIWC features help to clarify positive borderline cases by providing additional context and reducing ambiguity. Still, the number of label flips was very low, showing that the model's decisions were generally stable. The stabilizing effects may be explained by the ability of LIWC to reduce ambiguity in borderline cases. Prior research has shown that LIWC features:

Further evidence for the stabilizing role of LIWC comes from related work:

\begin{itemize}
    \item LIWC operationalizes psycholinguistic intentions by capturing affective, cognitive and social dimensions of language use~\cite{pennebaker2022liwc} and provides more reliable predictors than surface-level text features in personality modeling~\cite{farnadi2018user}. These stable markers remain invisible in purely linguistic features and can reduce ambiguity in borderline cases.
    \item In the context of cybergrooming, LIWC has been shown to clarify behaviors across different stages by highlighting psycholinguistic and discourse patterns~\cite{Cano2014} and to distinguish between authentic and deceptive relational intentions~\cite{broome2020psycholinguistic}. This could lead to a more contextualized understanding of grooming strategies.
    \item Beyond cybergrooming detection, LIWC has been shown to reduce variance across runs and stabilize predictions when being combined with embeddings~\cite{mehta2020bottomup}, leading to more consistent outcomes. This aligns with the reduction in label flips and the increased confidence observed in this thesis.
\end{itemize}

This suggests, that LIWC contributes to a stronger contextualization of conversations, lowering False Positives and enhancing the confidence of model predictions.
%%geändert

\section{LIWC as a Tool for Identifying Grooming and Non-Grooming Mechanisms}

The analysis of LIWC features in grooming and non-grooming chats on a global level, in chunks, and using SHAP highlights psycholinguistic patterns that are linked to existing research on cybergrooming. Particularly striking is the high proportion of the macro categorie \textit{Cognition} (together with\textit{discrepancy} and \textit{tentative language}), \textit{future focus}, \textit{home}, \textit{family} and \textit{affiliation} in the complete grooming conversations (Figure \ref{fig:liwc_global_analysis}). These findings reflect typical grooming narratives. Cano et al.~\cite{Cano2014} describe that in the trust development phase, references to \textit{home} and \textit{family} dominate, while the approach phase is characterized by planning markers like \textit{future} and motion semantics. Gupta et al.~\cite{gupta2012characterizingpedophileconversationsinternet} confirm this pattern using LIWC profiles from O’Connell’s~\cite{oconnell2003typology} six grooming phases, where the categories \textit{family} and \textit{home} act as markers of risk assessment, \textit{affiliation} indicates relationship building and \textit{sexual terms} represent the sexual phase. The global analyses also highlight the category \textit{discrepancy} (\textit{would/should/could}), which Cano et al.~\cite{Cano2014} and Gupta et al.~\cite{gupta2012characterizingpedophileconversationsinternet} describe as a marker for boundary testing and conditioning. The results therefore support existing phase and function models and at the same time show, that grooming conversations can be identified through a variety of overlapping markers. The same is demonstrated by Chiang and Grant~\cite{chiangandgrant2017online}, who identify 14 rhetorical moves that directly correspond to LIWC categories, including \textit{affiliation} (rapport building), \textit{discrepancy/tentative language} (boundary testing) and \textit{future} (planning). Furthermore, Lorenzo-Dus and Kinzel~\cite{LorenzoDus2019} and Powell et al.~\cite{powell2021online}  discuss that grooming in practice is not linear but rather “intermittent, without clear structure.” This explains the global analysis (Figure \ref{fig:liwc_global_analysis}), where all phase markers appear simultaneously, which reveals not a strict stage sequence but an overlapping profyile.

%%%geändert
The chunk-based analysis (Figure \ref{fig:liwc_chunked_analysis}) highlights that these markers appear not only at the level of complete conversations but already in short text segments of 512 tokens. This explains why transformer models achieve high detection performance at the chunk level, since local markers like \textit{(positive)emotion}, \textit{Allure}, \textit{Cognition} with \textit{cognitive processes}, \textit{Drives} with \textit{affiliation} and \textit{Social (behavior)} with \textit{politeness} are present there even if the difference is more subtle.


The SHAP analysis (Figure \ref{fig:feature_importance_by_class_combined}) provides a model based confirmation that the stylistic LIWC categories \textit{Tone}, \textit{Authentic}, \textit{Analytic}, and \textit{Clout}, as well as broader linguistic categories like  \textit{Affect}, \textit{Cognition} and \textit{Social processes} (each encompassing several subcategories), are key predictors for distinguishing grooming from non-grooming conversations. These findings also align with prior research on psycholinguistic markers in cybergrooming communication. Across the literature, \textbf{cognitive and social processes} emerge as central indicators of manipulative intent. Leiva-Bianchi et al.~\cite{leiva2024meta} showed the relevance of \textit{cognitive processes} (\textit{cogproc}, \textit{insight}, \textit{discrep}, \textit{tentat}), \textit{social markers} (\textit{affiliation}, \textit{family}, \textit{friend}), \textit{drives} (\textit{drives}, \textit{allure}), and \textit{emotional expressions} (\textit{affect}, \textit{emopos}, \textit{emoneg}), as well as sexual and politeness related language for grooming detection. Most of these categories appear among the top 20 SHAP features in the analysis in this thesis, with the macro group \textit{Cognition} dominating across all analyses, underscoring its important role in manipulative strategies. This cognitive focus is further complemented by \textbf{interpersonal and affective markers}. Broome et al.~\cite{broome2020psycholinguistic} found that groomers typically exhibit a characteristic combination of high \textit{Clout} (dominance/self-confidence), \textit{positive tone}, and moderate \textit{Authenticity}. The SHAP results prove these three dimensions as strongly weighted features, supporting their robustness as psycholinguistic markers. Broome et al. also identified \textit{cognitive}, \textit{social}, and \textit{present-focus} processes as key communicative patterns which are also mirrored here by the strong SHAP relevance of the features \textit{Cognition} and \textit{Social}. A similar set of linguistic signals was identified by Black et al.~\cite{black2015linguistic}, who grouped relevant markers into thematic clusters containing \textit{family/home} for risk assessment, \textit{pronouns/affiliation} for exclusivity, \textit{sexual/allure} for sexualization, and \textit{Cognition}, \textit{future}, \textit{emotion} and \textit{polite} for relationship work. These clusters are reflected in the present LIWC analyses where globally, \textit{family}, \textit{affiliation}, \textit{Social}, and \textit{Cognition} dominate and at the chunk level, \textit{affiliation} and \textit{Cognition} appear as strong predictors, while in the SHAP values, \textit{affiliation}, \textit{Cognition}, and \textit{Social} appear prominently. 
%%geändeert

\textbf{In addition to these category level observations, a closer review of the underlying LIWC subfeatures clarifies which linguistic dimensions most strongly drive the observed distinctions.} The stylistic features \textit{Analytic}, \textit{Clout}, \textit{Authentic}, and \textit{Tone} showed the highest model relevance, showing that overall stylistic tone and psychological stance are strong global indicators. Across all domains, \textbf{Cognition} had the most consistent influence, primarily through \textit{cognitive processes}, \textit{insight}, \textit{cause}, \textit{discrepancy}, and \textit{tentative language}. These subfeatures capture reasoning and uncertainty expressions that differentiate communicative patterns between both classes. \textbf{Affect} was driven by \textit{positive tone}, \textit{positive emotion} and \textit{swear}, while \textbf{Drives} were characterized by \textit{affiliation}. The \textbf{Social processes} domain was mainly shaped by \textit{social behavior}, \textit{politeness}, and \textit{female} references, which implies differences in interpersonal language. Additionally, the features \textbf{allure} and \textbf{sexual} stood out as individually relevant features, highlighting the thematic contrasts beyond the other aggregated categories. 

Finally, the \textbf{cybersecurity relevance} of these psycholinguistic categories has also been highlighted in recent work. Tshimula et al.~\cite{tshimula2024psychologicalprofilingcybersecuritylook} showed that linguistic and emotional cues are behavioral indicators of attackers, as LIWC captures increased self-focus, negative language, and cognitive process terms. This supports the SHAP relevance of the macro groups \textit{Cognition}, \textit{Clout}, and \textit{Affect}. 

Therefore, the SHAP analysis provides a confirmation of the established research on cybergrooming communication. \textbf{Overall, LIWC has proven in this work to be a very powerful tool for capturing thematic, linguistic and psychological aspects of grooming conversations, especially in the combination of global, chunk-based and model-driven analyses.} This is particularly evident in the ability of LIWC to detect subtle linguistic cues across different levels of analysis, underscoring its robustness in the context of cybergrooming detection.

%%geändert

\section{Efficiency potential through reduced feature selection}


The analysis of cumulative feature importance showed that around half of the features already account more than 80\% of the model's prediction. \textbf{This implies that there is a certain amount of redundancy in the feature set and that recognition performance could be maintained even with a reduced selection of features.} In practice, this suggests that models could focus on a smaller set of high-level LIWC dimensions that have shown the strongest discriminative power across analyses. Instead of including the full range of categories, a model could rely mainly on overarching stylistic categories like \textit{Tone}, \textit{Authenticity}, \textit{Analytic Thinking}, and \textit{Clout}, as well as the main psycholinguistic categories \textit{Affect}, \textit{Cognition} and \textit{Social Processes}. More linguistic features like \textit{Word Count}, \textit{Punctuation}, or \textit{Big Words}, could additionally serve as supplementary indicators rather than core features. In addition, a reduction in the number of features could also improve the interpretability with SHAP, since the exact calculation of Shapley values is associated with exponential computational effort depending on the number of features~\cite{fryer2021shapley}.
%%geändert
\section{Analysis of Misclassifications}

The analysis of misclassifications revealed an asymmetry between false positives and false negatives. False positives showed based on the LIWC features strong proximity to True Positives, where false negatives did not exhibit a clear pattern. This finding has strong implications for the precision–recall balance. \textbf{However, it should be noted, that the absolute number of misclassifications in this thesis was very low. Consequently, statistical findings regarding the proximity of false positives and false negatives should be interpreted with caution.}

In the full LIWC feature space, 87\% of False Positives were located closer to True Positives than to True Negatives. Even when restricting the analysis to the psychometric feature subset, the effect persisted, with approximately 66\% of False Positives clustering nearer to True Positives. Moreover, 44 LIWC features significantly distinguished False Positives from True Negatives with medium to large effect sizes in the full feature set. These results highlight, that False Positives are not random errors, but rather conversations that share core LIWC markers with real grooming interactions. The strong similarity of False Positives to True Positives explains, why high precision is difficult to achieve in cybergrooming detection. It is not a failure of the model, but a property of the domain, where non-grooming conversations can linguistically be more similar to grooming conversations. Consequently, future work should move beyond linguistic analysis and could incorporate contextual dimensions like user age, relationship context, platform specific cues and temporal development of conversations. Still, adding features from LIWC has been shown to reduce False Positives from the baseline model and improve precision, indicating that these features can help to clarify borderline cases.
%%%geändert
In contrast, False Negatives displayed no consistent proximity pattern. In the full LIWC feature space, only 46\% clustered closer to True Negatives and in the psychometric subset this proportion was 55\%. These values are close to chance level, indicating that False Negatives represent borderline cases without a clear proximity to either one class. This implies that the model tends to favor the negative class in ambiguous situations, which leads to missed detections of subtle grooming samples. The lack of a consistent LIWC profile for False Negatives makes them harder to analyse and address. To improve the classification of False Negatives, some methodological approaches can be considered. First, a feature fusion strategy could integrate additional contextual information like the relationship between interlocutors (for example familiar vs. unfamiliar) or the age of the authors, providing information that go beyond linguistic content. Second, adjusting the loss function by incorporating $F_{\beta}$ scores could allow the model to prioritize precision or recall depending on the application context. Third, SHAP based explainability offers insights by highlighting features where False Negatives mimic True Negatives. This could inform a re-weighting of categories, that are especially effective in distinguishing False Negatives from True Negatives and potentially reduce missed detections.  
%%%geändert

\section{Transparency and Ethical Implications}

When combining LIWC features with transformer representations, the resulting model gains a stronger performance and increases its interpretability. The use of SHAP explanations allows for a clear attribution of model decisions to specific psycholinguistic features, providing “explainable reasons” for the predictions. This not only increases the interpretability for researchers but also enhances trust among potential end users. LIWC features therefore give black-box models a “psychological grounding”, which makes their decisions more understandable and justifiable. Nevertheless, the main drawback of SHAP lies in its high computational cost, which limits the number of samples that can be analyzed. Therefore, alternative explainability methods like Integrated Gradients~\cite{integratedgradients} and Lime~\cite{ribeiro2016lime} could be explored as additional tools. These approaches may provide different perspectives on feature relevance, reduce computational overhead and together with SHAP give a more comprehensive explainability framework.
%%%geändert
\section{Broader Limitations and Future Directions}

So far, the discussion has focused on interpreting the main findings and their implications. However, it is also important to show broader limitations of this thesis and outline potential approaches for future research.  

A first set of limitations relates to the data. The PAN12 and PJ datasets are now more than a decade old, meaning that language, slang, and communication styles have naturally evolved since their collection. This temporal gap may limit the applicability of the findings to present grooming conversations. Moreover, it was necessary to strongly preprocess the data to handle slang. While this preprocessing step was necessary for an accurate LIWC feature extraction, it also creates a dependency on the quality of the applied data. In real world applications, where new slang and abbreviations regularly arise, maintaining such a pipeline would be challenging and may require alternative strategies like embedding-based approaches that can quickly adapt to unseen words in context.  

Another limitation concerns the linguistic scope of the data. Both datasets are written in English, and the LIWC features were derived from an English dictionary. Since LIWC categories are strongly connected to specific lexical items, the findings may not generalize across other language contexts. Still, cybergrooming occurs worldwide across diverse linguistic communities, which underscores the need for multilingual datasets to enable the development of universally applicable detection models. 
%%geändert
When looking at the generalizability of the findings of this thesis further, a core limitation of the Perverted Justice dataset lies in the use of decoy victims. These conversations often involve adult volunteers posing as minors, which produces linguistic styles and response behaviors that differ from child victims. As highlighted in prior work~\cite{chiangandgrant2017online}, this raises concerns about the authenticity of the data, potentially limiting the real world applicability of research findings. Similarly, Broome et al.~\cite{broome2020psycholinguistic} emphasize that reliance on decoy victims introduces unnatural conversational dynamics, undermining the general validity of the dataset. Since this thesis relies strongly on linguistic features, some categories like \textit{sexual} or \textit{affiliation} may be over or underrepresented in the PJ data. For example, decoys may appear more cooperative and responsive, increasing the frequency of affiliation markers, while real victims may exhibit stronger resistance and emotional distress, which would influence the appearance of affective terms.  

In addition, the dataset used for model training in this thesis was purposely balanced to ensure SHAP analyses. While this was necessary for methodological reasons, this does not reflect the real world distribution of grooming conversations, which are way less frequent than non-grooming conversations. This unnatural balance may lead to an overestimation of model performance, particularly in terms of precision and recall. Future research should therefore assess models on more realistic and imbalanced datasets to better compare their practical utility. The lack of positive examples also includes a broader structural challenge in the field. The lack of publicly available and sufficiently large cybergrooming datasets complicates the training of robust models and increases the risk of domain leakage if the training and test data are not strictly separated.  
%%geändert
Finally, certain methodological limitations should be adressed. The reliance on BERT-based models with a 512-token limit meant that not all conversations could be fully preserved, even when chunking was applied, leading to a potential loss of context. Furthermore, the fixed training of three epochs without early stopping does not rule out overfitting, even if the model was trained with increased dropout and label smoothing to mitigate overconfidence. What limits this concern is, that the relative gains from LIWC fusion over the BERT baseline were consistent across epochs 1–3. Still, a stricter control would include a small dev set with early stopping, reporting train/dev learning curves and complementing single-split results with group-stratified $k$-fold cross-validation and out-of-domain evaluation. More strategies like learning-curve monitoring, group-stratified cross-validation, and out-of-domain evaluations could also strengthen generalization in future work.  

\textbf{Together, these limitations point towards a need for future research.}

First, future research should explore model architectures that go beyond the 512-token constraint of BERT. Hierarchical architectures and long-context transformer models would enable the processing of entire conversations and all grooming phases rather than truncated segments, thereby preserving important contextual cues~\cite{vogt2021early}. Another promising application lies in the early detection of grooming. Instead of focusing on full conversations, models could be trained to identify grooming behavior at earlier stages based on LIWC, which would be crucial for timely intervention~\cite{vogt2021early}. These approaches could also investigate which LIWC features are present in different phases of grooming, combining psycholinguistic analysis with machine learning models. However, this would require datasets that explicitly encode conversational phases (for example ChatCoder2, which was developed by McGhee et al.~\cite{chatcoder}). 
%%%geändert
Secondly, including augmentation techniques like paraphrasing or backtranslation could be analyzed according to the effect on LIWC distributions, explainability analysis and model robustness. 

Also, future research could experiment with different models and fusion strategies. As already mentioned, stronger baselines like RoBERTa~\cite{liu2019roberta} or DeBERTa v3~\cite{he2023debertav3} may outperform BERT, and it would be interesting to test whether integrating LIWC features show improvements in these architectures as well. At the same time, care should be taken since such models are trained on broad corpora that may already contain parts of datasets like PAN12. Beyond transformers, feature fusion with other neural architectures like LSTMs or CNNs could offer deeper insights into how linguistic and psychometric features complement each other with the goal of cybergrooming detection.  

Finally, new research directions could be developed around the role of LIWC features themselves. For example, models could be designed to predict grooming phases based only on LIWC categories, or to evaluate the predictive power of reduced feature sets by training models on only the most informative LIWC categories. Applying the proposed approach directly to the PAN12 dataset would also allow for more direct comparability with previous work and exclude potential data balance effects. However, this would again require an extensive slang handling process to maximize the extraction of relevant LIWC features.

But most important of all:
 
\textbf{Progress in grooming detection will require larger, more diverse, and multilingual datasets, ideally including ethically sourced real victim data. Although this introduces ethical, legal, and privacy challenges, collaborations with law enforcement agencies and child protection organizations may offer options to anonymized datasets that maintain compliance with ethical standards. In parallel, future studies should explore computationally efficient methods for feature selection and dimensionality reduction, architectures capable of modeling longer conversational sequences, and adaptive pipelines that can cope with evolving online language. Addressing these challenges will be the key to improving the robustness and ecological validity of grooming detection models.}  
%%%geändert







